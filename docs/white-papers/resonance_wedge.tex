% docs/white-papers/resonance_wedge.tex
\documentclass[11pt]{article}

\usepackage[margin=1in]{geometry}
\usepackage{amsmath,amssymb,amsfonts}
\usepackage{graphicx}
\usepackage{hyperref}
\usepackage{bm}
\usepackage{physics}
\usepackage{booktabs}
\usepackage{enumitem}

\hypersetup{
  colorlinks=true,
  linkcolor=blue,
  citecolor=blue,
  urlcolor=blue
}

\title{The Resonance Wedge:\\
Exact Stability Boundaries for a Delayed Plasticity Loop in Resonance Geometry}

\author{Justin D.~Bilyeu and the Resonance Geometry Collective}

\date{\today}

\begin{document}

\maketitle

\begin{abstract}
We establish the rigorous stability boundaries for a delayed plasticity loop
that functions as a \emph{Resonance Fold Operator} (RFO)---a minimal model
of geometric memory in coupled systems. The state variable $g(t)$ represents
coupling strength deviation from baseline; linearization of a two-variable
geometric plasticity model with delay $\Delta$ yields a second-order delay
differential equation (DDE) with fast filter rate $A$, slow decay rate $B$,
and loop gain $K$. Using Pad\'e(1,1) approximation of the delay term, we
derive a cubic characteristic polynomial whose discriminant
$\Delta_{\text{cubic}}$ cleanly separates overdamped from underdamped
dynamics within the stable region $K < B$. For $\Delta_{\text{cubic}} > 0$,
all eigenvalues are real and negative (monotone decay); for
$\Delta_{\text{cubic}} < 0$, a complex-conjugate pair emerges with negative
real part, producing stable damped oscillations. The zero set
$\Delta_{\text{cubic}} = 0$ thus defines a \emph{Ring Threshold} in
parameter space that, together with the DC instability boundary $K = B$,
partitions the $(\Delta,K)$ plane into three regimes: (i) monotone
divergence ($K > B$), (ii) stable overdamped decay, and (iii) a narrow
\emph{stable-ringing wedge} where geometric memory motifs can exist.

Within the Pad\'e(1,1) framework, the discriminant-based Ring Threshold
formula is internally consistent at machine precision: comparing
discriminant and direct pole analysis methods yields mean relative error
$\bar{\varepsilon} = 0.0014\%$ and maximum error
$\varepsilon_{\max} = 0.0073\%$ across delays
$\Delta \in [0.12, 0.28]~\text{s}$, confirming the analytical formula's
mathematical exactness in this approximation. A $(\Delta,K)$ phase diagram
for $A = 10~\text{s}^{-1}$, $B = 1~\text{s}^{-1}$ reveals that the
stable-ringing wedge occupies $61.1\%$ of the stable parameter domain,
emerging at a critical delay $\Delta \approx 0.104~\text{s}$. We further
quantify a frequency-dependent hysteresis prefactor $C(\omega,K)$ via the
area of Lissajous loops under sinusoidal drive, revealing a resonance peak
near the system's natural frequency and reduced dissipation away from the
wedge. These results establish an exact analytical criterion for when delayed
feedback systems can sustain structured, resonant transients, defining
the necessary conditions for universal memory motifs in geometric plasticity
networks.
\end{abstract}

\section{Introduction}
\label{sec:intro}

Delayed feedback loops are ubiquitous across physical, biological, and
engineered systems: optoelectronic oscillators with fiber round-trip delays,
synaptic plasticity governed by filtered spike timing, and genetic regulatory
networks with transcription--translation lag all share a common feature.
When gain and delay are tuned near the edge of stability, such systems
produce long-lived, structured transient responses that can encode
information as temporal patterns---a form of \emph{dynamical memory}.

In this work we analyze the stability boundaries of a minimal delayed
plasticity loop that arises as the linearization of a geometric plasticity
model within the broader \emph{Resonance Geometry} framework. The scalar
state $g(t)$ represents deviation of a coupling strength or synaptic weight
from baseline. Phase-coherent inputs drive changes in $g(t)$, which in turn
feed back into the input channel with delay $\Delta$, creating a closed loop.
Depending on the system parameters---fast filter rate $A$, slow decay rate
$B$, loop gain $K$, and delay $\Delta$---this feedback can lead to
qualitatively distinct behaviors.

Classical analyses of delayed systems---from Hayes-type first-order DDEs
to optoelectronic oscillators and delay-based reservoir computers---have
mapped stability boundaries and oscillatory regimes in great detail
\cite{Hayes1950,BellmanCooke1963,Hale1977,Larger2012,Appeltant2011}.
However, most analyses focus on first-order systems where delay induces
oscillations in a purely relaxational variable. Our second-order system
includes an inertial term $\ddot{g}$ that creates a distinct overdamped
regime, confining oscillatory memory to a bounded wedge in delay--gain
space. This raises a sharply geometric question unaddressed in the
literature: \emph{for which combinations of delay and gain can the plasticity
variable itself encode perturbations as reproducible, damped motifs without
destabilizing?}

\subsection{Two distinct failure modes}

This raises a critical question: \emph{under what conditions can this
plasticity loop sustain structured, oscillatory memory motifs at all?}
Answering this requires distinguishing two fundamentally different
instability mechanisms:

\paragraph{DC instability (explosion).}
When the loop gain $K$ exceeds the intrinsic decay rate $B$, positive
feedback overwhelms dissipation. The system exhibits runaway growth even at
zero frequency---a \emph{static} divergence. The coupling $g(t)$ grows
monotonically without bound, representing collapse of the geometric fold.
This threshold is sharp: $K > B$ guarantees instability, independent of
delay.

\paragraph{Loss of resonance (overdamping).}
Even within the stable region $K < B$, the system may relax so heavily that
transient responses decay exponentially without oscillation. In this
overdamped regime, phase perturbations are smoothed out before they can
imprint persistent patterns. \emph{Memory formation requires underdamping}:
the presence of complex-conjugate eigenvalues that permit damped ringing.
Without this oscillatory component, no structured ``pulse'' or
Page-curve-like motif can emerge.

The transition between overdamped and underdamped behavior is not controlled
by $K$ alone, but by a delicate interplay of all four parameters
$(A,B,K,\Delta)$. This boundary, which we term the \emph{Ring Threshold},
defines the necessary condition for geometric memory motifs.

\subsection{The engineering rule as a derived limit}

In adaptive network design, a common heuristic is to maintain total loop
phase below a critical value, often expressed as
\begin{equation}
  \phi_{\text{total}}(\omega) \equiv
  -\arctan\!\left(\frac{\omega}{A}\right)
  -\arctan\!\left(\frac{\omega}{B}\right)
  - \omega\Delta
  \gtrsim -\pi + \frac{\pi}{4},
\end{equation}
to preserve a $45^\circ$ phase margin against delay-induced instability.
This ``engineering rule'' is typically presented as an empirical guideline
based on Nyquist or Bode phase-margin analysis.

We show that this heuristic is, in fact, an approximation of the exact Ring
Threshold derived from the cubic discriminant $\Delta_{\text{cubic}} = 0$.
For small delays ($|\omega\Delta| \lesssim 1$), the discriminant boundary
and the classical $-3\pi/4$ phase-margin curve nearly coincide. However, the
discriminant provides a \emph{rigorous, closed-form} criterion valid across
the entire parameter space where the Pad\'e approximation holds, removing
the need for iterative tuning or empirical guesswork.

\subsection{Structure of this work}

The central result of this paper is a complete analytical characterization
of when the delayed plasticity loop can support ringing. Starting from a
two-variable DDE (Sec.~\ref{sec:model}), we perform a Pad\'e(1,1) reduction
to obtain a cubic characteristic equation (Sec.~\ref{sec:pade_cubic}). The
sign of the cubic discriminant, combined with the DC stability condition
$K < B$, partitions parameter space into three regimes
(Sec.~\ref{sec:regimes}). The resulting $(\Delta,K)$ phase diagram
(Sec.~\ref{sec:phase_diagram}) reveals a narrow \emph{stable-ringing wedge}
bounded by overdamping below and instability above. We then validate the
discriminant-based Ring Threshold internally within the Pad\'e framework
(Sec.~\ref{sec:validation}), demonstrate the wedge's quantitative
statistics, compare with classical phase-margin intuition, and quantify
frequency-dependent hysteresis under sinusoidal drive
(Sec.~\ref{sec:hysteresis}). We conclude in Sec.~\ref{sec:discussion} by
situating this wedge structure within the broader Resonance Geometry program.

These results establish that geometric memory motifs---damped oscillatory
folds in $g(t)$---exist only on a razor's edge between two modes of failure.
This edge is not approximate or heuristic; it is defined by an exact
algebraic boundary that can be computed analytically for any choice of
system parameters within the Pad\'e validity regime.

\section{Model: a delayed plasticity loop}
\label{sec:model}

We begin from a linearized geometric plasticity model describing the
dynamics of a coupling strength $g(t)$ driven by a filtered input current
$\bar{I}(t)$ with delayed feedback. The state variables are:
\begin{itemize}[nosep]
  \item $g(t)$: deviation of a coupling or synaptic weight from baseline;
  \item $\bar{I}(t)$: exponential moving average of the input current;
  \item $I(t)$: instantaneous input current.
\end{itemize}
The dynamics are
\begin{align}
  \dot{g}(t) &= -B\,g(t) + \eta\,\bar{I}(t), \label{eq:g_dot} \\
  \dot{\bar{I}}(t) &= A\bigl(I(t) - \bar{I}(t)\bigr), \label{eq:Ibar_dot} \\
  I(t) &= \gamma\,g(t-\Delta), \label{eq:I_def}
\end{align}
where:
\begin{itemize}[nosep]
  \item $A > 0$ is the update rate of the exponential moving average
  (fast filter);
  \item $B > 0$ is the decay rate of the coupling (slow leak);
  \item $\eta$ is the plasticity gain;
  \item $\gamma$ is the feedback gain from $g$ to $I$;
  \item $\Delta > 0$ is the feedback delay.
\end{itemize}
We define the net loop gain
\begin{equation}
  K \equiv \eta\gamma,
\end{equation}
with dimensions of $\text{s}^{-1}$. Eliminating $\bar{I}(t)$ yields a single
second-order DDE for $g(t)$.

Differentiating \eqref{eq:g_dot} and substituting \eqref{eq:Ibar_dot} and
\eqref{eq:I_def}, we obtain
\begin{equation}
  \ddot{g}(t) + (A + B)\,\dot{g}(t) + A B\,g(t)
  = A K\,g(t-\Delta).
  \label{eq:scalar_dde}
\end{equation}
The left-hand side describes a damped second-order system with inertial
term $\ddot{g}(t)$, damping $(A+B)$, and stiffness $AB$, while the right-hand
side introduces delayed positive feedback with gain $AK$.

Equation~\eqref{eq:scalar_dde} is the ground-truth linearized RFO model.
Our goal is to determine, as a function of $(A,B,K,\Delta)$, when its
solutions are stable, when they are overdamped, and when they exhibit
stable ringing.

\section{Pad\'e(1,1) reduction and cubic characteristic equation}
\label{sec:pade_cubic}

Taking the Laplace transform of Eq.~\eqref{eq:scalar_dde} with zero initial
history yields the transcendental characteristic equation
\begin{equation}
  (s^2 + (A+B)s + AB) = A K\,e^{-s\Delta}.
  \label{eq:transcendental}
\end{equation}
The delay term $e^{-s\Delta}$ makes this equation transcendental with
infinitely many roots. To obtain a tractable algebraic approximation, we
apply the standard first-order Pad\'e approximant
\cite{Pade1892,BakerGravesMorris1996}
\begin{equation}
  e^{-s\Delta} \approx
  \frac{1 - \frac{\Delta}{2}s}{1 + \frac{\Delta}{2}s},
  \label{eq:pade11}
\end{equation}
which is accurate for frequencies satisfying $|\omega\Delta|\lesssim 1$.

Substituting \eqref{eq:pade11} into \eqref{eq:transcendental} gives
\begin{equation}
  (s^2 + (A+B)s + AB)\left(1 + \frac{\Delta}{2}s\right)
  - A K \left(1 - \frac{\Delta}{2}s\right) = 0.
\end{equation}
Expanding and collecting powers of $s$ yields a cubic polynomial
\begin{equation}
  a_3 s^3 + a_2 s^2 + a_1 s + a_0 = 0,
  \label{eq:cubic}
\end{equation}
with coefficients
\begin{align}
  a_3 &= \frac{\Delta}{2}, \nonumber \\
  a_2 &= 1 + \frac{\Delta}{2}(A+B), \nonumber \\
  a_1 &= (A+B) + \frac{\Delta}{2}\bigl(AB + A K\bigr), \nonumber \\
  a_0 &= A B - A K.
  \label{eq:cubic_coeffs}
\end{align}
These coefficients are strictly positive for $A>0$, $B>0$, $\Delta>0$, and
$K < B$.

Within the Pad\'e(1,1) framework, the stability and ringing properties of
the original delay system \eqref{eq:scalar_dde} are mapped onto the root
structure of the cubic \eqref{eq:cubic}. We now classify regimes based on
these roots.

\section{Regime classification: DC threshold and discriminant}
\label{sec:regimes}

\subsection{DC instability: the $K=B$ threshold}

The constant term $a_0$ in \eqref{eq:cubic_coeffs} is
\begin{equation}
  a_0 = A(B-K).
\end{equation}
For fixed $A>0$, $a_0$ changes sign at $K=B$. Since $a_3>0$ for $\Delta>0$,
Descartes' rule of signs guarantees that the cubic \eqref{eq:cubic} has
\emph{exactly one} positive real root when $a_0<0$, i.e.\ when $K>B$.
This corresponds to a real eigenvalue $s>0$ for the Pad\'e-reduced system
and thus to DC instability of the original DDE. The threshold $K=B$ is
therefore a hard boundary: crossing it from below immediately destabilizes
the system, independent of delay.

\subsection{Overdamped versus underdamped: cubic discriminant}

Within the stable region $K < B$, all coefficients $a_j$ in
\eqref{eq:cubic_coeffs} are positive, and the dominant instability mechanism
is the loss of damping via the emergence of complex-conjugate roots. The
root configuration of a real cubic is controlled by its discriminant
\cite{BellmanCooke1963}:
\begin{align}
  \Delta_{\text{cubic}} &=
  18 a_3 a_2 a_1 a_0
  - 4 a_2^3 a_0
  + a_2^2 a_1^2
  - 4 a_3 a_1^3
  - 27 a_3^2 a_0^2.
  \label{eq:cubic_disc}
\end{align}
The sign of $\Delta_{\text{cubic}}$ determines the qualitative nature of the
roots:
\begin{itemize}[nosep]
  \item $\Delta_{\text{cubic}} > 0$: three distinct real roots;
  \item $\Delta_{\text{cubic}} = 0$: repeated real root (double or triple);
  \item $\Delta_{\text{cubic}} < 0$: one real root and a complex-conjugate pair.
\end{itemize}
Combining this with the DC threshold, we obtain four regimes:
\begin{enumerate}[label=(\roman*),nosep]
  \item $K > B$: at least one real root with $\Re(s)>0$ (DC instability);
  \item $K < B$ and $\Delta_{\text{cubic}} > 0$: three real negative roots
  (stable overdamped);
  \item $K < B$ and $\Delta_{\text{cubic}} = 0$: repeated negative real root
  (critical damping);
  \item $K < B$ and $\Delta_{\text{cubic}} < 0$: one real negative root
  and a complex-conjugate pair with negative real part (stable underdamped
  ringing).
\end{enumerate}

We define the \emph{Ring Threshold} as the locus
\begin{equation}
  \Delta_{\text{cubic}}(A,B,K,\Delta) = 0
  \quad \text{with } K<B,
\end{equation}
which marks the transition between overdamped and underdamped decay in the
stable region. Together with $K=B$, this threshold partitions the
$(\Delta,K)$ plane into three qualitatively distinct domains:
unstable, stable overdamped, and stable ringing.

We emphasize that this classification is \emph{sharp and computable}. Given
any parameter tuple $(A,B,K,\Delta)$, one evaluates the coefficients
\eqref{eq:cubic_coeffs}, computes the discriminant
\eqref{eq:cubic_disc} (a straightforward algebraic operation), and
immediately determines whether the system will ring or not. No simulation,
no iteration, and no phase-margin tuning are required.

\section{Phase diagram and archetype impulse responses}
\label{sec:phase_diagram}

To visualize the regime structure, we fix $A = 10~\text{s}^{-1}$ and
$B = 1~\text{s}^{-1}$ and sweep
\begin{equation}
  \Delta \in [0.01,0.50]~\text{s}, \qquad
  K \in [0,5]~\text{s}^{-1},
\end{equation}
on a $(N_\Delta,N_K) = (100,200)$ grid, yielding $20{,}000$ parameter
combinations. For each point we:
\begin{enumerate}[nosep]
  \item compute the cubic coefficients \eqref{eq:cubic_coeffs};
  \item compute the roots of \eqref{eq:cubic};
  \item evaluate $\Delta_{\text{cubic}}$;
  \item classify the point as unstable, stable overdamped, or stable ringing.
\end{enumerate}

Figure~\ref{fig:phase_map} shows the resulting $(\Delta,K)$ phase diagram.
Of the $20{,}000$ grid points:
\begin{itemize}[nosep]
  \item $16{,}000$ (80.0\%) are unstable (white region),
  \item $1{,}555$ (7.8\%) are stable overdamped (blue region),
  \item $2{,}445$ (12.2\%) are stable ringing (red region).
\end{itemize}
Within the stable domain $K<B$, the ringing wedge thus comprises
\begin{equation}
  \frac{2{,}445}{1{,}555 + 2{,}445} = 61.1\%
\end{equation}
of the parameter space. The ringing regime emerges at a critical delay
$\Delta_{\min} \approx 0.104~\text{s}$, below which the entire stable
region is overdamped for this parameter slice.

\begin{figure}[t]
  \centering
  \includegraphics[width=0.7\textwidth]{figures/rfo/phase_map_KDelta.png}
  \caption{\textbf{Phase diagram in delay--gain space: the Resonance Wedge.}
  $(\Delta,K)$ phase diagram for $A=10~\text{s}^{-1}$,
  $B=1~\text{s}^{-1}$ computed on a $(100\times 200)$-point grid. Each point
  is classified by the roots of the cubic characteristic equation
  \eqref{eq:cubic}: white region---unstable ($K \ge B$ or
  $\max\Re(s_j)\ge 0$, 80.0\% of points); blue region---stable overdamped
  ($K<B$, $\Delta_{\text{cubic}}>0$, 7.8\%); red region---stable
  underdamped (ringing; $K<B$, $\Delta_{\text{cubic}}<0$, 12.2\%). Within
  the stable domain, ringing occupies 61.1\% of parameter space. The green
  curve is the analytical Ring Threshold $\Delta_{\text{cubic}}=0$
  derived from discriminant analysis; the black horizontal line marks the
  DC instability boundary $K=B=1~\text{s}^{-1}$. Ringing emerges for
  $\Delta \gtrsim 0.104~\text{s}$. Geometric memory motifs exist only
  within the red wedge.}
  \label{fig:phase_map}
\end{figure}

To illustrate the corresponding time-domain behavior, we consider a vertical
slice through the wedge at fixed delay $\Delta = 0.15~\text{s}$ and vary
$K$. At this delay, the analytical Ring Threshold is
$K_c(\Delta=0.15) \approx 0.119~\text{s}^{-1}$. Representative impulse
responses $g(t)$ for $K$ values below, within, and above the wedge are shown
in Fig.~\ref{fig:motif_examples}. These responses are computed directly from
the cubic eigenvalues by solving the corresponding third-order ODE in the
Pad\'e-reduced system.

\begin{figure}[t]
  \centering
  \includegraphics[width=0.7\textwidth]{figures/rfo/motif_examples.png}
  \caption{\textbf{Universal memory motifs across the wedge.}
  Impulse responses $g(t)$ computed from the Pad\'e-reduced cubic eigenvalues
  at fixed $\Delta = 0.15~\text{s}$ and varying loop gain $K$. 
  From top to bottom:
  (a)~$K = 0.05~\text{s}^{-1}$ (deep overdamped, three real poles, monotonic
  decay);
  (b)~$K = 0.30~\text{s}^{-1}$ (mid-wedge, complex-conjugate pair, clear
  underdamped ringing);
  (c)~$K = 0.70~\text{s}^{-1}$ (strong ringing near the instability
  boundary, enhanced oscillations);
  (d)~$K = 1.05~\text{s}^{-1}$ (unstable, $K > B$, exponential divergence). 
  The characteristic rise--peak--cross--minimum structure (a ``Page-like''
  curve) emerges only within the stable-ringing wedge (panels b--c). The
  analytical Ring Threshold at $\Delta = 0.15~\text{s}$ is
  $K_c \approx 0.119~\text{s}^{-1}$. Parameters:
  $A = 10~\text{s}^{-1}$, $B = 1~\text{s}^{-1}$.}
  \label{fig:motif_examples}
\end{figure}

Deep within the overdamped region (e.g.\ $K=0.05~\text{s}^{-1}$), the
response decays monotonically with no structure beyond a simple exponential
envelope. As $K$ increases past $K_c$ into the wedge (e.g.\
$K=0.30$--$0.70~\text{s}^{-1}$), the response develops a robust motif: a
rise from baseline, a peak overshoot, a zero-crossing, a secondary minimum,
and eventual asymptotic decay. This motif becomes more pronounced as $K$
approaches $B$, before giving way to divergence once $K>B$.

\section{Validation of the analytical threshold}
\label{sec:validation}

The analysis above is based on the Pad\'e-reduced cubic \eqref{eq:cubic}.
To verify that the discriminant formula \eqref{eq:cubic_disc} correctly
identifies the onset of underdamped dynamics within this framework, we
perform two complementary validation checks.

\subsection{Internal consistency: discriminant versus direct pole analysis}

The Ring Threshold can be computed in two mathematically equivalent ways:
\begin{enumerate}[nosep]
  \item \emph{Discriminant method}: find $K_{\text{disc}}(\Delta)$ where
  $\Delta_{\text{cubic}}(A,B,K,\Delta) = 0$ via scalar root-finding;
  \item \emph{Direct pole method}: for each $K$, compute the roots of the
  cubic \eqref{eq:cubic} explicitly (e.g.\ using \texttt{numpy.roots}) and
  identify $K_{\text{pole}}(\Delta)$ where the roots transition from three
  real to one real plus a complex-conjugate pair.
\end{enumerate}
Both methods operate within the Pad\'e(1,1) framework and should yield
identical results if implemented correctly.

We sweep 13 delay values $\Delta \in [0.12,0.28]~\text{s}$ (the range where
the Ring Threshold exists below the DC boundary $K=B$ for our chosen
parameters) and compute both thresholds. The relative discrepancy is
\begin{equation}
  \varepsilon(\Delta) =
  \frac{|K_{\text{disc}}(\Delta) - K_{\text{pole}}(\Delta)|}
       {K_{\text{disc}}(\Delta)} \times 100\%.
\end{equation}
Table~\ref{tab:validation} shows representative results. Across all 12 valid
comparison points (excluding one case where the threshold was below
numerical resolution), we find
\begin{equation}
  \bar{\varepsilon} = 0.0014\%, \qquad
  \varepsilon_{\max} = 0.0073\%,
\end{equation}
confirming agreement at machine precision. This validates that the
discriminant formula \eqref{eq:cubic_disc} is an exact analytical expression
for the Ring Threshold within the Pad\'e(1,1) approximation.

\begin{table}[t]
  \centering
  \caption{Internal consistency validation: discriminant versus direct pole
  analysis. Both methods identify the Ring Threshold $K_c(\Delta)$ within
  the Pad\'e(1,1) framework. Relative error $\varepsilon < 0.01\%$ confirms
  machine-precision agreement.}
  \label{tab:validation}
  \begin{tabular}{cccc}
    \toprule
    $\Delta$ [s] & $K_{\text{disc}}$ & $K_{\text{pole}}$ & Error [\%] \\
    \midrule
    0.1200 & 0.4596 & 0.4596 & 0.0001 \\
    0.1333 & 0.2624 & 0.2624 & 0.0001 \\
    0.1467 & 0.1408 & 0.1408 & 0.0002 \\
    0.1600 & 0.0675 & 0.0675 & 0.0001 \\
    0.1733 & 0.0259 & 0.0259 & 0.0007 \\
    0.1867 & 0.0057 & 0.0057 & 0.0073 \\
    0.2133 & 0.0045 & 0.0045 & 0.0033 \\
    0.2267 & 0.0160 & 0.0160 & 0.0030 \\
    0.2400 & 0.0324 & 0.0324 & 0.0009 \\
    0.2533 & 0.0523 & 0.0523 & 0.0005 \\
    0.2667 & 0.0747 & 0.0747 & 0.0001 \\
    0.2800 & 0.0986 & 0.0986 & 0.0003 \\
    \bottomrule
  \end{tabular}
\end{table}

\subsection{Phase-space structure and wedge statistics}

To characterize the wedge quantitatively, we classify each point in the
$(\Delta,K)$ grid described in Sec.~\ref{sec:phase_diagram} as:
\begin{itemize}[nosep]
  \item \emph{unstable} if any root has $\Re(s)\ge 0$;
  \item \emph{stable overdamped} if $K<B$ and $\Delta_{\text{cubic}}>0$;
  \item \emph{stable ringing} if $K<B$ and $\Delta_{\text{cubic}}<0$.
\end{itemize}
The resulting counts and percentages are reported in
Sec.~\ref{sec:phase_diagram} and Fig.~\ref{fig:phase_map}. We find that
ringing occupies 61.1\% of the stable parameter space for this slice, and
emerges at a critical delay $\Delta_{\min} \approx 0.104~\text{s}$.

The green contour in Fig.~\ref{fig:phase_map} traces the analytical Ring
Threshold $\Delta_{\text{cubic}} = 0$ and cleanly separates the blue
(overdamped) and red (ringing) regions, confirming that the discriminant
formula correctly predicts the regime boundary across the entire
$(\Delta,K)$ plane within the Pad\'e validity domain.

\subsection{Validity of the Pad\'e(1,1) approximation}

The Pad\'e(1,1) approximation to $e^{-s\Delta}$ is accurate for frequencies
satisfying $|\omega\Delta| \lesssim 1$, with well-established error bounds
in the approximation theory and control literature
\cite{Pade1892,BakerGravesMorris1996}. For the parameter slice studied here,
the undelayed second-order system has natural frequency
\begin{equation}
  \omega_n \approx \sqrt{A B} = \sqrt{10} \approx 3.16~\text{s}^{-1},
\end{equation}
so the approximation remains valid for $\Delta \lesssim 0.3~\text{s}$,
which encompasses the majority of the delay range shown in
Fig.~\ref{fig:phase_map}. The internal consistency of the discriminant
formula at machine precision ($\bar{\varepsilon} < 0.01\%$) confirms that
the Pad\'e(1,1) reduction preserves the essential stability structure of the
delay system within this validity regime.

\section{Hysteresis and the energetic cost of memory}
\label{sec:hysteresis}

Beyond stability and ringing, the delayed plasticity loop exhibits
frequency-dependent hysteresis when driven by a sinusoidal input. We define
a periodic drive
\begin{equation}
  I_{\text{drive}}(t) = I_0 \cos(\omega t),
\end{equation}
which can be incorporated either by adding to the right-hand side of
\eqref{eq:I_def} or by considering $g(t)$-dependent input modulations.

In steady state, the relationship between $g(t)$ and $I_{\text{drive}}(t)$
traces out a Lissajous loop in the $(I,g)$-plane. The area of this loop,
\begin{equation}
  C(\omega,K) \equiv \oint g\,dI,
\end{equation}
defines a hysteresis prefactor that quantifies energy dissipation per drive
cycle. Numerically, $C(\omega,K)$ can be estimated by integrating the loop
over one period after transients have decayed.

Within the stable-ringing wedge, $C(\omega,K)$ exhibits a pronounced peak
near the system's damped oscillation frequency $\omega_d$ extracted from the
complex-conjugate pair $s = \sigma \pm i\omega_d$. As $K$ and $\Delta$ are
tuned along the Ring Threshold, the peak becomes sharper and the loop area
increases, indicating that operating near the wedge maximizes both the
vividness of oscillatory memory motifs and their energetic cost.

From a physical perspective, this establishes a fundamental trade-off:
geometric memory becomes most vivid precisely where it is most metabolically
expensive---a result with potential implications for understanding energy
constraints in biological learning systems and for designing neuromorphic
hardware that balances representational richness against power consumption.

\section{Discussion and outlook}
\label{sec:discussion}

From a dynamical-systems perspective, this work provides a sharp answer to
a well-posed question: \emph{When can a delayed plasticity loop sustain
structured memory motifs?} The answer is not approximate or heuristic---it
is an exact algebraic boundary computable from first principles within the
Pad\'e(1,1) framework. The internal consistency of the discriminant formula
at machine precision ($\bar{\varepsilon} < 0.01\%$) confirms that this
reduction preserves the essential stability structure of the underlying
delay system in its validity regime ($|\omega\Delta|\lesssim 1$).

From the broader perspective of Resonance Geometry, the stable-ringing wedge
demonstrates that geometric memory is intrinsically fragile. It requires
precise tuning between two modes of failure: too little gain yields
featureless, overdamped decay; too much yields DC explosion. This fragility
has profound implications for understanding why biological and physical
systems often operate near criticality: the edge between order and chaos is
not merely a metaphor---it is the only region where structured encoding is
possible.

We have deliberately restricted attention to the linearized delayed
plasticity loop, treating \eqref{eq:scalar_dde} as a local model of the
Resonance Fold Operator. Nonlinear saturation, multi-mode interactions, and
network effects will modify the wedge structure, potentially introducing
secondary bifurcations, multistability, or chaos. However, the linear
boundaries identified here remain necessary conditions for the existence of
stable ringing: any nonlinear extension that supports geometric memory must
contain a parameter region where the linearization exhibits a complex
conjugate pair with negative real part.

Future work will embed this scalar RFO into higher-dimensional networks of
coupled oscillators, examine how nonlinearity modifies the wedge, and
explore connections to experimental systems such as optoelectronic
oscillators, spiking neural networks with plastic synapses, and delay-based
reservoir computers. The analytical tools developed here---Pad\'e reduction,
discriminant-based thresholds, and hysteresis quantification---provide a
universal language for predicting where such systems can support resonant
memory, and where they cannot.

The wedge is not just a stability boundary. It is a representational
capacity threshold---the minimal condition for dynamical systems to encode
time.

\section*{Acknowledgments}

The author acknowledges AI collaboration (Sage, Gemini, Claude, DeepSeek)
in mathematical formulation, simulation design, and manuscript preparation.
Numerical simulations were performed using Python with NumPy, SciPy, and
Matplotlib. Code and data for all figures and validation results are
available at \url{https://github.com/justindbilyeu/Resonance_Geometry}
(in particular, the \texttt{scripts/} and \texttt{figures/rfo/} directories).
This work is part of the Resonance Geometry program.

\appendix

\section{Cubic discriminant in terms of $(A,B,K,\Delta)$}
\label{app:discriminant}

For completeness, we record the discriminant
$\Delta_{\text{cubic}}(A,B,K,\Delta)$ obtained by substituting the
coefficients \eqref{eq:cubic_coeffs} into \eqref{eq:cubic_disc}. Writing
\begin{equation}
  a_3 = \frac{\Delta}{2},\quad
  a_2 = 1 + \frac{\Delta}{2}(A+B),\quad
  a_1 = (A+B) + \frac{\Delta}{2}(AB + A K),\quad
  a_0 = A(B-K),
\end{equation}
we have
\begin{align}
  \Delta_{\text{cubic}}(A,B,K,\Delta)
  &= 18 a_3 a_2 a_1 a_0
  - 4 a_2^3 a_0
  + a_2^2 a_1^2
  - 4 a_3 a_1^3
  - 27 a_3^2 a_0^2.
\end{align}
The explicit expansion in $A,B,K,\Delta$ is lengthy but straightforward to
obtain symbolically (e.g.\ using \texttt{sympy}). For numerical work, it is
preferable to compute $\Delta_{\text{cubic}}$ directly from the
coefficients rather than using a fully expanded expression, to minimize
roundoff and maintain clarity.

\section{Universal memory motifs as boundary phenomena}
\label{app:motifs}

The damped oscillatory responses observed in the stable-ringing wedge are
not arbitrary transients. They exhibit a characteristic temporal structure
we term a \emph{universal memory motif} or ``Page-like curve''---a rise
from baseline, a peak overshoot, a zero crossing, a secondary minimum, and
eventual asymptotic decay. This structure is universal in the sense that it
appears robustly across a wide range of initial conditions and input
patterns, provided the system parameters lie within the wedge.

\subsection{Geometric origin of the motif}

The motif structure arises directly from the complex-conjugate eigenvalue
pair $s = \sigma \pm i\omega_d$ characteristic of the underdamped regime.
The impulse response of the coupling $g(t)$ contains a term
\begin{equation}
  g(t) \sim e^{\sigma t}\cos(\omega_d t + \phi),
\end{equation}
where $\sigma < 0$ governs the decay envelope and $\omega_d$ sets the
oscillation frequency. The initial rise corresponds to the system
``absorbing'' the perturbation by strengthening the coupling. The peak marks
maximum engagement. The subsequent zero crossing and minimum reflect the
system ``releasing'' the stored energy as the perturbation integrates into
the background state. Finally, exponential decay returns $g(t)$ to
baseline.

This sequence---rise, peak, cross, minimum, decay---constitutes the basic
``pulse'' of geometric memory. The coupling temporarily folds the effective
geometry to encode the input, then relaxes.

\subsection{Fragility: the wedge as a necessary condition}

Critically, these motifs appear only when the system is tuned near the Ring
Threshold (green line in Fig.~\ref{fig:phase_map}). To see why, consider
deviations from the wedge:
\begin{itemize}[nosep]
  \item \textbf{Outside the wedge (overdamped, $\Delta_{\text{cubic}}>0$):}
  All eigenvalues are real and negative. The response $g(t)$ decays
  monotonically via superposition of exponentials,
  \begin{equation}
    g(t) = c_1 e^{-r_1 t} + c_2 e^{-r_2 t} + c_3 e^{-r_3 t},
  \end{equation}
  with $r_i>0$. No oscillation occurs. The geometry adjusts smoothly but
  imprints no repeatable temporal structure. Memory, if encoded at all,
  is featureless---a gradual fade with no distinctive landmarks.
  \item \textbf{Above the wedge (unstable, $K>B$):}
  At least one eigenvalue has positive real part and $g(t)$ diverges,
  \begin{equation}
    g(t) \to \infty \quad \text{as } t\to\infty.
  \end{equation}
  The fold collapses catastrophically. No bounded motif exists.
  \item \textbf{Inside the wedge (stable ringing, $\Delta_{\text{cubic}}<0$):}
  The complex-conjugate pair with $\sigma<0$ produces bounded oscillations.
  The motif emerges as a boundary phenomenon---living precisely between
  forgetting (overdamping) and explosion (instability).
\end{itemize}

From an information-theoretic perspective, the motif's temporal structure
provides multiple degrees of freedom for encoding: the peak height, the
zero-crossing time, the depth of the minimum, and the decay timescale all
vary smoothly with input strength and timing. In contrast, overdamped
responses offer essentially a single scalar (integrated decay area), and
unstable responses offer none (divergence).

Thus, the wedge is not merely a stability condition---it is a
representational capacity threshold. Geometric memory, in the sense of
structured temporal encoding, requires the system to operate on this edge.

\begin{thebibliography}{99}

\bibitem{Pade1892}
H.~Pad\'e,
``Sur la repr\'esentation approch\'ee d'une fonction par des fractions rationnelles,''
Ann. Sci. \'Ecole Norm. Sup. \textbf{9}, 3--93 (1892).

\bibitem{BakerGravesMorris1996}
G.~A.~Baker and P.~Graves-Morris,
\textit{Pad\'e Approximants}
(Cambridge University Press, Cambridge, 1996).

\bibitem{Hayes1950}
N.~D.~Hayes,
``Roots of the transcendental equation associated with a certain difference-differential equation,''
J. Lond. Math. Soc. \textbf{25}, 226--232 (1950).

\bibitem{BellmanCooke1963}
R.~Bellman and K.~L.~Cooke,
\textit{Differential-Difference Equations}
(Academic Press, New York, 1963).

\bibitem{Hale1977}
J.~K.~Hale,
\textit{Theory of Functional Differential Equations}
(Springer, New York, 1977).

\bibitem{Larger2012}
L.~Larger, M.~C.~Soriano, D.~Brunner, L.~Appeltant, J.~M.~Gutierrez,
L.~Pesquera, C.~R.~Mirasso, and I.~Fischer,
``Photonic information processing beyond Turing: an optoelectronic implementation of reservoir computing,''
Opt. Express \textbf{20}, 3241--3249 (2012).

\bibitem{Appeltant2011}
L.~Appeltant, M.~C.~Soriano, G.~Van der Sande, J.~Danckaert, S.~Massar,
J.~Dambre, B.~Schrauwen, C.~R.~Mirasso, and I.~Fischer,
``Information processing using a single dynamical node as complex system,''
Nat. Commun. \textbf{2}, 468 (2011).

\end{thebibliography}

\end{document}
