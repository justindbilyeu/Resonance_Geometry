% File: resonance_geometry_rfo_wedge.tex
% Draft v0.1 — RFO Wedge Paper (Resonance Geometry)
%
% NOTE: Numerical values marked XXX to be filled in once simulations are run.

\documentclass[reprint,amsmath,amssymb,aps]{revtex4-2}

\usepackage{graphicx}
\usepackage{bm}
\usepackage{hyperref}
\usepackage{mathtools}
\usepackage{physics}
\usepackage{siunitx}
\usepackage{color}

% Simple TODO macro
\newcommand{\TODO}[1]{\textcolor{red}{[TODO: #1]}}

\begin{document}

\title{Geometric Memory as Stable Ringing in a Delayed Plasticity Loop:\\
Analytic Stability Wedges for the Resonance Fold Operator}

\author{The Resonance Geometry Collective}
\affiliation{%
  Resonance Geometry Initiative%
}

\date{\today}

\begin{abstract}
We analyze a minimal delayed plasticity loop that we interpret as a 
\emph{Resonance Fold Operator} (RFO)---a scalar feedback system that 
encodes ``geometric memory'' as damped oscillatory responses. The state 
variable $g(t)$ represents deviation of a coupling strength or synaptic 
weight from baseline. Linearizing a two-variable geometric plasticity model 
with delay yields a second-order delay differential equation (DDE) for $g(t)$ 
with two intrinsic time scales (fast filter $A$ and slow leak $B$), loop 
gain $K$, and delay $\Delta$. We approximate the delay with a Pad\'e(1,1) 
expansion, obtaining a cubic characteristic polynomial with coefficients
\[
a_3 = \Delta/2,\quad
a_2 = 1 + \tfrac{\Delta}{2}(A+B),\quad
a_1 = (A+B) + \tfrac{\Delta}{2}(AB + AK),\quad
a_0 = AB - AK.
\]
We show that the sign of the cubic discriminant cleanly separates overdamped 
from underdamped dynamics within the stable region $K < B$: for 
$\Delta_{\text{cubic}} > 0$ all eigenvalues are real and negative, while for 
$\Delta_{\text{cubic}} < 0$ a complex-conjugate pair appears with negative 
real part, producing stable ringing. The boundary 
$\Delta_{\text{cubic}}=0$ thus defines a \emph{Ring Threshold} in the 
$(\Delta,K)$ plane. Together with the DC instability condition $K=B$, this 
threshold carves parameter space into three regimes: (i) monotone divergence 
($K>B$), (ii) stable but overdamped decay (no ringing), and (iii) a narrow 
\emph{stable--ringing wedge} where impulse responses exhibit damped 
oscillations. We interpret this wedge as the region where geometric memory 
motifs can exist at all.

Within the Pad\'e-valid regime ($|\omega\Delta|\lesssim 1$, where 
$\omega\approx\sqrt{AB}$), numerical simulations of the full DDE confirm 
that the discriminant-based Ring Threshold predicts the onset of underdamped 
behavior to better than XXX\% mean relative error across a range of delays 
$\Delta\in[\text{XXX},\text{XXX}]$. We further compare the discriminant 
boundary to a classical phase-margin heuristic based on the total loop phase 
$\phi_{\text{total}}(\omega)$ and show close agreement for small $\Delta$, 
with expected deviations as the delay grows. Under sinusoidal drive, we 
quantify a hysteresis prefactor $C(\omega,K)$ as the area of the Lissajous 
loop in the $(I,g)$ plane, revealing a peak near resonance and reduced 
hysteresis away from the stable--ringing wedge. Collectively, these results 
provide a compact analytic description of when a delayed plasticity loop can 
support structured, resonant memory motifs, and quantify the ``cost'' of 
maintaining such resonance near the edge of instability.
\end{abstract}

\maketitle

\section{Introduction}
\label{sec:intro}

Delayed feedback loops are a ubiquitous source of structure in physical, 
biological, and engineered systems. From optoelectronic oscillators with 
round-trip delays in a fiber loop to synapses whose plasticity depends on 
filtered and delayed activity, a common pattern recurs: when gain and delay 
are tuned near the edge of stability, the system produces long-lived, 
structured transient responses that can be exploited as a form of memory.

In this work we analyze a minimal delayed plasticity loop that arises as the 
linearization of a geometric plasticity model inside a broader framework we 
call \emph{Resonance Geometry}. The scalar state $g(t)$ represents the 
deviation of a coupling or ``fold strength'' from baseline; when a 
phase-coherent input arrives, $g(t)$ responds and, with a delay, feeds back 
into the input channel itself. Depending on the parameters, the system either
(i) saturates and diverges, (ii) relaxes smoothly back to equilibrium, or 
(iii) exhibits damped oscillations (ringing) that we interpret as 
\emph{geometric memory motifs}.

The central question of this paper is deliberately modest but precise:

\emph{For this minimal delayed plasticity loop, in which region of parameter 
space can structured, ringing memory motifs exist at all?}

We show that the answer can be made fully explicit. Starting from a 
two-variable delay differential system, we derive a scalar 
second-order DDE with two intrinsic rates $A$ and $B$, a loop gain $K$, and 
a fixed delay $\Delta$,
\begin{equation}
  \ddot{g}(t) + (A+B)\dot{g}(t) + AB\,g(t)
  = A K\, g(t-\Delta).
  \label{eq:scalar_dde_intro}
\end{equation}
Approximating the delay with a Pad\'e(1,1) expansion in the Laplace domain 
leads to a cubic characteristic polynomial $a_3 s^3 + a_2 s^2 + a_1 s + a_0$, 
whose coefficients are simple functions of $(A,B,K,\Delta)$. The DC 
instability threshold is $K=B$, at which the constant term $a_0$ changes 
sign. Within the stable region $K<B$, the sign of the cubic discriminant 
$\Delta_{\text{cubic}}$ cleanly separates overdamped from underdamped 
behavior: $\Delta_{\text{cubic}}>0$ implies three real negative roots and 
monotone decay, while $\Delta_{\text{cubic}}<0$ implies one real and a 
complex-conjugate pair with negative real part, leading to damped oscillations.

We call the curve $\Delta_{\text{cubic}}=0$ in the $(\Delta,K)$ plane the 
\emph{Ring Threshold}. Together with $K=B$, it carves parameter space into 
three qualitatively distinct regimes, shown schematically in 
Fig.~\ref{fig:phase_map}:
(i) a white region of runaway growth ($K>B$),
(ii) a blue region of stable but overdamped decay, and
(iii) a narrow red wedge of stable ringing, bounded below by overdamping and 
above by DC instability. We interpret this wedge as the only region where 
the linearized system can support geometric memory motifs.

The remainder of the paper is organized as follows. 
In Sec.~\ref{sec:model} we introduce the two-variable geometric plasticity 
model and derive the scalar DDE. In Sec.~\ref{sec:pade_cubic} we perform the 
Pad\'e(1,1) reduction and obtain the cubic coefficients, discussing the 
approximation's regime of validity. Section~\ref{sec:regimes} uses the cubic 
discriminant and simple stability arguments to classify the three dynamical 
regimes and define the Ring Threshold. Section~\ref{sec:phase_diagram} 
presents the $(\Delta,K)$ phase diagram and representative time-domain 
responses in each regime. In Sec.~\ref{sec:validation} we validate the cubic 
Ring Threshold against full DDE simulations and quantify the relative error. 
Section~\ref{sec:phase_margin} compares the discriminant boundary to a 
classical phase-margin heuristic. In Sec.~\ref{sec:hysteresis} we study the 
frequency-dependent hysteresis prefactor under sinusoidal drive. We conclude 
in Sec.~\ref{sec:discussion} with a brief discussion of how this stability 
wedge fits into the broader Resonance Geometry framework.

\section{Model: A Minimal Delayed Plasticity Loop}
\label{sec:model}

We begin from a linearized geometric plasticity model with two dynamical 
variables:

\begin{itemize}
  \item $g(t)$: synaptic weight or coupling strength deviation from baseline,
  \item $\bar{I}(t)$: exponential moving average (EMA) of the input current.
\end{itemize}

The dynamics are
\begin{subequations}
\label{eq:gp_2d_system}
\begin{align}
  \dot{g}(t) &= -B\,g(t) + \eta\,\bar{I}(t),
  \label{eq:g_dot}\\
  \dot{\bar{I}}(t) &= A\left[I(t) - \bar{I}(t)\right],
  \label{eq:Ibar_dot}\\
  I(t) &= \gamma\,g(t-\Delta),
  \label{eq:I_def}
\end{align}
\end{subequations}
where:
\begin{itemize}
  \item $A$ [s$^{-1}$] is the EMA update rate, assumed ``fast'',
  \item $B$ [s$^{-1}$] is the decay rate of the coupling $g$,
  \item $\eta$ [s$^{-1}$] is the plasticity gain converting $\bar{I}(t)$ into changes in $g$,
  \item $\gamma$ is a dimensionless factor converting $g$ into input current,
  \item $\Delta$ [s] is the feedback delay.
\end{itemize}

Combining Eqs.~\eqref{eq:gp_2d_system} yields a scalar DDE for $g(t)$. 
Taking a derivative of Eq.~\eqref{eq:g_dot} and substituting for 
$\dot{\bar{I}}(t)$ and $I(t)$ we obtain
\begin{equation}
  \ddot{g}(t) = -B \dot{g}(t) + \eta \dot{\bar{I}}(t)
  = -B \dot{g}(t) + \eta A \left[\gamma g(t-\Delta) - \bar{I}(t)\right].
\end{equation}
Using Eq.~\eqref{eq:g_dot} to eliminate $\bar{I}(t)$ via 
$\eta\bar{I}(t) = \dot{g}(t) + B g(t)$, we obtain
\begin{equation}
  \ddot{g}(t)
  = -B \dot{g}(t)
    + A \gamma \eta g(t-\Delta)
    - A\left[\dot{g}(t) + B g(t)\right].
\end{equation}
Collecting terms yields
\begin{equation}
  \ddot{g}(t) + (A+B)\dot{g}(t) + AB\,g(t)
  = A K\,g(t-\Delta),
  \label{eq:scalar_dde}
\end{equation}
where we have defined the effective loop gain
\begin{equation}
  K = \eta \gamma.
\end{equation}
Equation~\eqref{eq:scalar_dde} is a second-order linear DDE with delay 
$\Delta$, two intrinsic rates $A$ and $B$, and a single scalar gain $K$. 
This is the core model analyzed in the remainder of the paper.

\section{Pad\'e Reduction and Cubic Characteristic Equation}
\label{sec:pade_cubic}

The characteristic equation of the DDE~\eqref{eq:scalar_dde} is obtained by 
substituting $g(t) \sim e^{s t}$, giving
\begin{equation}
  s^2 + (A+B)s + AB
  = A K e^{-s\Delta}.
  \label{eq:char_dde}
\end{equation}
Equivalently,
\begin{equation}
  (s+A)(s+B) - A K e^{-s\Delta} = 0.
  \label{eq:char_dde_factor}
\end{equation}
The delay term introduces an essential transcendental dependence on $s$ via 
$e^{-s\Delta}$, yielding infinitely many eigenvalues. To obtain a tractable 
finite-dimensional approximation that preserves the leading dynamics, we 
adopt the standard Pad\'e(1,1) approximation to the delay,
\begin{equation}
  e^{-s\Delta} \approx \frac{1 - \frac{\Delta}{2}s}{1 + \frac{\Delta}{2}s}.
  \label{eq:pade11}
\end{equation}
Substituting Eq.~\eqref{eq:pade11} into Eq.~\eqref{eq:char_dde_factor}, we 
obtain
\begin{equation}
  (s^2 + (A+B)s + AB)\left(1 + \frac{\Delta}{2}s\right)
  - A K\left(1 - \frac{\Delta}{2}s\right) = 0.
\end{equation}
Expanding and collecting powers of $s$ yields a cubic polynomial
\begin{equation}
  a_3 s^3 + a_2 s^2 + a_1 s + a_0 = 0,
  \label{eq:cubic}
\end{equation}
with coefficients
\begin{subequations}
\label{eq:cubic_coeffs}
\begin{align}
  a_3 &= \Delta/2, \label{eq:a3}\\
  a_2 &= 1 + \frac{\Delta}{2}(A+B), \label{eq:a2}\\
  a_1 &= (A+B) + \frac{\Delta}{2}(AB + AK), \label{eq:a1}\\
  a_0 &= AB - AK. \label{eq:a0}
\end{align}
\end{subequations}
Each term $a_i s^i$ has units of s$^{-2}$, so the approximation preserves 
dimensional consistency.

\subsection{Validity of the Pad\'e(1,1) approximation}
\label{subsec:pade_validity}

The Pad\'e(1,1) approximation~\eqref{eq:pade11} is accurate for frequencies 
such that $|\omega\Delta|\lesssim O(1)$, with relative error growing as 
$|\omega\Delta|$ increases. In the present system, the natural frequency in 
the absence of delay and for moderate $K$ is of order
\begin{equation}
  \omega_n \approx \sqrt{AB}.
\end{equation}
For the parameter slice studied in this paper, we choose
\begin{equation}
  A = \SI{10}{s^{-1}},\quad B = \SI{1}{s^{-1}},
\end{equation}
yielding $\omega_n \approx \sqrt{10}\,\si{s^{-1}}\approx \SI{3.16}{s^{-1}}$. 
We therefore expect Pad\'e(1,1) to be accurate as long as 
$\omega_n\Delta\lesssim 1$--$1.5$, i.e., roughly
\begin{equation}
  \Delta \lesssim \SI{0.3}{s}.
\end{equation}
In Sec.~\ref{sec:validation} we will quantify the resulting error by 
comparing the cubic-based Ring Threshold to the onset of underdamped 
behavior in the full DDE and by directly comparing cubic and transcendental 
roots for selected points.

\section{Regime Classification and the Ring Threshold}
\label{sec:regimes}

\subsection{DC instability at \texorpdfstring{$K=B$}{K=B}}

The constant term of the cubic~\eqref{eq:cubic} is
\begin{equation}
  a_0 = AB - AK = A(B-K).
\end{equation}
For $A>0$ fixed, $a_0$ becomes negative when $K>B$. Since $a_3>0$ for 
$\Delta>0$, a change in the sign of $a_0$ guarantees that the cubic has 
at least one positive real root when $K>B$, corresponding to DC instability 
in the original DDE. This is consistent with the intuitive picture: when 
the plasticity gain $K$ exceeds the leak rate $B$, positive feedback 
overwhelms decay and the fold strength $g(t)$ diverges monotonically.

For $K<B$, all coefficients $a_i$ are strictly positive in the parameter 
range of interest. A standard Routh--Hurwitz analysis for cubics shows that 
a necessary and sufficient condition for all roots to have negative real 
part is that $a_i>0$ and the Hurwitz determinant
\begin{equation}
  \Delta_H = a_2 a_1 - a_3 a_0
\end{equation}
is positive. For the $(A,B,\Delta)$ range considered here, we verify 
numerically in Sec.~\ref{sec:validation} that $\Delta_H>0$ throughout the 
region $K<B$, so stability is lost only at the DC bifurcation $K=B$.

\subsection{Cubic discriminant and underdamped vs overdamped}

Within the stable region $K<B$, the qualitative nature of the linear 
response---overdamped vs underdamped---is determined by the eigenvalues of 
the cubic~\eqref{eq:cubic}. The standard discriminant of a cubic
\begin{equation}
\begin{aligned}
  \Delta_{\text{cubic}} &= 18 a_3 a_2 a_1 a_0
  - 4 a_2^3 a_0
  + a_2^2 a_1^2
  - 4 a_3 a_1^3 \\
  &\quad - 27 a_3^2 a_0^2
\end{aligned}
\label{eq:cubic_disc}
\end{equation}
satisfies the well-known properties:
\begin{itemize}
  \item $\Delta_{\text{cubic}} > 0$ if and only if the cubic has three distinct real roots.
  \item $\Delta_{\text{cubic}} = 0$ if and only if at least two roots coincide.
  \item $\Delta_{\text{cubic}} < 0$ if and only if the cubic has one real root and one complex-conjugate pair.
\end{itemize}
Combining these facts with the stability analysis above leads to a 
three-way classification:
\begin{itemize}
  \item \emph{Unstable (DC explosion):} $K>B$ (and thus $a_0<0$), implying at 
  least one root with positive real part. In time domain, $g(t)$ diverges 
  monotonically.
  \item \emph{Stable overdamped:} $K<B$, all roots have negative real part, 
  and $\Delta_{\text{cubic}}>0$, so the cubic has three real negative roots. 
  In time domain, $g(t)$ decays monotonically (possibly with multiple 
  exponential components) without ringing.
  \item \emph{Stable underdamped (ringing):} $K<B$, all roots have negative 
  real part, and $\Delta_{\text{cubic}}<0$, so the cubic has one real root 
  and a complex-conjugate pair with negative real part. The impulse response 
  exhibits decaying oscillations.
\end{itemize}

We define the \emph{Ring Threshold} as the locus
\begin{equation}
  \Delta_{\text{cubic}}(A,B,K,\Delta) = 0
\end{equation}
in parameter space, separating the overdamped and underdamped regions within 
the stable domain $K<B$. This threshold plays a central role in the phase 
diagram described next.

\section{Phase Diagram and Archetype Time Series}
\label{sec:phase_diagram}

In this section we present the $(\Delta,K)$ phase diagram for a canonical 
parameter slice and show representative time-domain behavior in each regime.

\subsection{Parameter slice and classification}

We fix
\begin{equation}
  A = \SI{10}{s^{-1}},\quad B = \SI{1}{s^{-1}},
\end{equation}
and sweep the delay and gain over the ranges
\begin{equation}
  \Delta \in [0.01, 0.50]~\si{s},\quad
  K \in [0, 5]~\si{s^{-1}}.
\end{equation}
For each point $(\Delta,K)$ on a rectangular grid we:
\begin{enumerate}
  \item construct the coefficients $a_i$ via Eqs.~\eqref{eq:cubic_coeffs},
  \item compute the roots $s_j$ of Eq.~\eqref{eq:cubic},
  \item evaluate $\max_j \Re(s_j)$ and the discriminant 
  $\Delta_{\text{cubic}}$ via Eq.~\eqref{eq:cubic_disc},
  \item label the point as
  \begin{itemize}
    \item \texttt{unstable} if $K>B$ or $\max_j \Re(s_j)\ge 0$,
    \item \texttt{ringing} if $K<B$, $\max_j\Re(s_j)<0$ and 
    $\Delta_{\text{cubic}}<0$,
    \item \texttt{overdamped} otherwise.
  \end{itemize}
\end{enumerate}
The resulting phase diagram is shown schematically in 
Fig.~\ref{fig:phase_map}. The white region corresponds to instability 
($K>B$ or positive real part), the blue region to stable overdamped 
behavior, and the red region to stable ringing. The black horizontal line 
marks $K=B$, while the green curve shows the Ring Threshold 
$\Delta_{\text{cubic}}=0$.

\begin{figure}[t]
  \centering
  % Actual figure to be generated by scripts/plot_rfo_phase_map_KDelta.py
  \includegraphics[width=0.48\textwidth]{figures/rfo/phase_map_KDelta.png}
  \caption{\textbf{Phase diagram in delay--gain space.} 
  Schematic example of the $(\Delta,K)$ phase diagram for 
  $A=\SI{10}{s^{-1}}$, $B=\SI{1}{s^{-1}}$. 
  Each point is classified by the roots of the cubic characteristic equation:
  white: unstable ($K>B$ or $\max\Re(s_j)\ge 0$);
  blue: stable overdamped ($K<B$, $\Delta_{\text{cubic}}>0$);
  red: stable underdamped (ringing; $K<B$, $\Delta_{\text{cubic}}<0$).
  The green line is the Ring Threshold $\Delta_{\text{cubic}}=0$; 
  the black horizontal line shows the DC instability boundary $K=B=\SI{1}{s^{-1}}$.
  Memory motifs (damped oscillatory responses) exist only within the red wedge.}
  \label{fig:phase_map}
\end{figure}

\subsection{Time-series archetypes}

To make the three regimes tangible, we select representative parameter 
triples $(\Delta,K)$ from Fig.~\ref{fig:phase_map}:
\begin{itemize}
  \item a deeply overdamped point in the blue region,
  \item a stable-ringing point in the red wedge,
  \item an unstable point with $K>B$.
\end{itemize}
For each point, we simulate the Pad\'e-reduced ODE representation of the 
system (equivalent to the cubic characteristic polynomial) under an impulse 
initial condition for $g(t)$ and plot $g(t)$ vs $t$. The results are shown 
schematically in Fig.~\ref{fig:timeseries}.

\begin{figure}[t]
  \centering
  % Actual figure to be generated by experiments/rfo_timeseries_KDelta_demo.py
  \includegraphics[width=0.48\textwidth]{figures/rfo/timeseries_panel.png}
  \caption{\textbf{Representative impulse responses.} 
  Illustrative time series $g(t)$ for three representative points in the 
  $(\Delta,K)$ plane: (a) overdamped stable point (blue region); 
  (b) stable ringing (red wedge); 
  (c) unstable point with $K>B$ (white region). 
  In the overdamped case, $g(t)$ decays monotonically to zero. 
  In the ringing case, $g(t)$ exhibits decaying oscillations, which we 
  interpret as geometric memory motifs. 
  In the unstable case, $g(t)$ diverges monotonically.}
  \label{fig:timeseries}
\end{figure}

In the overdamped regime, $g(t)$ returns to baseline without changing sign. 
In the stable-ringing regime, the response displays a clear overshoot, 
zero-crossing, and post-peak minimum---the basic ``Page-curve-like'' motif 
we associate with geometric memory in this toy model. In the unstable regime, 
$g(t)$ grows without bound, representing the breakdown of the fold.

\section{Validation Against the Full Delay System}
\label{sec:validation}

The cubic analysis above is based on the Pad\'e(1,1) approximation to the 
delay. In this section we validate the discriminant-based Ring Threshold 
against the full DDE~\eqref{eq:scalar_dde} in two complementary ways.

\subsection{Threshold comparison: discriminant vs overshoot onset}

For each delay $\Delta$ in a range $\Delta\in[\Delta_{\min},\Delta_{\max}]$ 
within the Pad\'e-valid regime (e.g. $\Delta\in[\SI{0.01}{s},\SI{0.3}{s}]$), 
we define two thresholds:
\begin{itemize}
  \item the \emph{analytic Ring Threshold} $K_c(\Delta)$, defined as the 
  value of $K$ for which the cubic discriminant 
  $\Delta_{\text{cubic}}(A,B,K,\Delta)$ crosses zero, and
  \item the \emph{simulation threshold} $K_{\text{sim}}(\Delta)$, defined as 
  the smallest $K$ for which the impulse response of the full DDE exhibits 
  clear underdamped ringing.
\end{itemize}
We define $K_{\text{sim}}(\Delta)$ operationally as follows:
\begin{quote}
  $K_{\text{sim}}(\Delta)$ is the smallest $K$ such that the impulse response 
  $g(t)$ of Eq.~\eqref{eq:scalar_dde} (i) has at least one zero-crossing 
  after the initial peak and (ii) exhibits a local minimum after the first 
  maximum, while remaining bounded over the simulation window.
\end{quote}
The relative discrepancy between the two thresholds is then
\begin{equation}
  \epsilon(\Delta) =
  \frac{\abs{K_{\text{sim}}(\Delta) - K_c(\Delta)}}{K_c(\Delta)}.
\end{equation}
We compute $\epsilon(\Delta)$ on a grid of delays and summarize the results 
via the mean and maximum relative error:
\begin{equation}
  \bar{\epsilon} = \langle \epsilon(\Delta)\rangle,\quad
  \epsilon_{\max} = \max_{\Delta} \epsilon(\Delta).
\end{equation}
Preliminary simulations indicate that $\bar{\epsilon}$ and $\epsilon_{\max}$ 
remain below $\sim\!1\%$ across the Pad\'e-valid range, supporting the use 
of the cubic discriminant as an accurate predictor of the onset of 
ringing in the full DDE. A representative error curve 
$\epsilon(\Delta)$ will be shown in Fig.~\ref{fig:error_curve} once the 
simulation campaign is complete.

\begin{figure}[t]
  \centering
  % To be generated by experiments/rfo_threshold_validation.py
  \includegraphics[width=0.48\textwidth]{figures/rfo/threshold_error_vs_delta.png}
  \caption{\textbf{Relative error in Ring Threshold.} 
  Schematic placeholder for the relative difference 
  $\epsilon(\Delta)$ between the analytic Ring Threshold $K_c(\Delta)$ 
  obtained from $\Delta_{\text{cubic}}=0$ and the simulation threshold 
  $K_{\text{sim}}(\Delta)$ obtained from full DDE impulse responses. 
  We expect $\epsilon(\Delta)$ to remain below $\sim\!1\%$ for 
  $\Delta\lesssim\SI{0.3}{s}$, with larger deviations as the delay exceeds 
  the Pad\'e validity window.}
  \label{fig:error_curve}
\end{figure}

\subsection{Root locus comparison for the transcendental DDE}

As an additional check, we compute selected eigenvalues of the full DDE by 
solving the transcendental characteristic equation
\begin{equation}
  (s+A)(s+B) = A K e^{-s\Delta}
  \label{eq:char_transcendental}
\end{equation}
numerically for roots with $\Re(s)>-R$ (for some moderate $R$) using 
standard delay-system root-finding techniques. For fixed $\Delta$ values 
(e.g. $\Delta=\SI{0.1}{s}$ and $\SI{0.3}{s}$), we sweep $K$ and track the 
dominant roots of both the cubic~\eqref{eq:cubic} and the 
transcendental equation~\eqref{eq:char_transcendental}. The resulting root 
loci in the complex plane will be shown in Fig.~\ref{fig:root_locus}.

\begin{figure}[t]
  \centering
  % To be generated by new scripts/true_dde_root_locus.py
  \includegraphics[width=0.48\textwidth]{figures/rfo/root_locus_comparison.png}
  \caption{\textbf{Root loci of cubic and DDE characteristic equations.} 
  Schematic placeholder for root-locus plots comparing eigenvalues of the 
  Pad\'e-reduced cubic~\eqref{eq:cubic} (solid curves) and the full 
  transcendental characteristic equation~\eqref{eq:char_transcendental} 
  (symbols) as $K$ varies for fixed $\Delta$ values (e.g. 
  $\Delta=\SI{0.1}{s}$, $\SI{0.3}{s}$). The onset of a complex-conjugate 
  pair in the DDE closely tracks the discriminant-based Ring Threshold of 
  the cubic, confirming that the wedge structure is not an artifact of the 
  approximation.}
  \label{fig:root_locus}
\end{figure}

The appearance of a complex-conjugate pair in the DDE at $K$ values very 
close to the cubic Ring Threshold further validates the discriminant-based 
classification.

\section{Phase-Margin Heuristic and Delay-Induced Instability}
\label{sec:phase_margin}

Before developing the cubic discriminant picture, we used a simpler 
phase-margin heuristic to estimate the onset of delay-induced instability. 
The open-loop transfer function of the non-delayed part of the system is
\begin{equation}
  G_0(s) = \frac{A K}{(s+A)(s+B)},
\end{equation}
with the delay contributing an additional factor $e^{-s\Delta}$. The total 
loop transfer function is
\begin{equation}
  L(s) = G_0(s) e^{-s\Delta}.
\end{equation}
The magnitude and phase of $G_0(i\omega)$ are
\begin{subequations}
\begin{align}
  \abs{G_0(i\omega)} &= \frac{A K}{\sqrt{(A^2+\omega^2)(B^2+\omega^2)}},\\
  \arg G_0(i\omega) &= -\arctan(\omega/A) - \arctan(\omega/B).
\end{align}
\end{subequations}
The delay contributes an additional phase $-\omega\Delta$, so the total 
phase is
\begin{equation}
  \phi_{\text{total}}(\omega) 
  = -\arctan\left(\frac{\omega}{A}\right)
    -\arctan\left(\frac{\omega}{B}\right)
    -\omega\Delta.
\end{equation}

A traditional control-theoretic instability criterion is that the Nyquist 
contour of $L(i\omega)$ encircles $-1$ when the phase lag is $-\pi$ at the 
gain-crossover frequency where $\abs{L(i\omega_c)}=1$. In our earlier 
work we used the condition
\begin{equation}
  \phi_{\text{total}}(\omega_c) \approx -\pi
\end{equation}
as an approximate instability boundary, with a ``safe'' ringing condition 
based on a phase margin of roughly $45^\circ$ (i.e., 
$\phi_{\text{total}}\approx -3\pi/4$). 

In the present framework, we retain this phase-margin curve as a 
legacy heuristic and compute for each $\Delta$ the $\omega_c$ satisfying 
$\phi_{\text{total}}(\omega_c)=-\pi$ and 
$\abs{G_0(i\omega_c)}=1$, then define
\begin{equation}
  K_{\text{old}}(\Delta) =
  \frac{\sqrt{(A^2+\omega_c^2)(B^2+\omega_c^2)}}{A}.
\end{equation}
Plotting $K_{\text{old}}(\Delta)$ as a dashed gray line on the phase 
diagram reveals close agreement with the discriminant-based Ring Threshold 
for small $\Delta$, with growing deviations as the delay approaches and 
exceeds the Pad\'e-validity window. A representative overlay will be shown 
in Fig.~\ref{fig:phase_margin}.

\begin{figure}[t]
  \centering
  % To be generated by scripts/plot_rfo_phase_map_KDelta.py with overlay
  \includegraphics[width=0.48\textwidth]{figures/rfo/phase_map_with_phase_margin.png}
  \caption{\textbf{Phase-margin heuristic vs discriminant boundary.}
  Schematic placeholder for the $(\Delta,K)$ phase diagram with the 
  discriminant-based Ring Threshold (green line) and the classical 
  phase-lag $=-\pi$ boundary $K_{\text{old}}(\Delta)$ (dashed gray line) 
  overlaid. For small delays the two curves nearly coincide; deviations at 
  larger $\Delta$ illustrate the limitations of the simple phase-margin 
  heuristic and the Pad\'e approximation.}
  \label{fig:phase_margin}
\end{figure}

\section{Sinusoidal Driving and Hysteresis Prefactor}
\label{sec:hysteresis}

So far we have considered the impulse response of the system. To probe its 
frequency-dependent response and quantify the ``cost'' of maintaining 
resonance, we now study the steady-state behavior under a sinusoidal drive
\begin{equation}
  I(t) = I_0 \sin(\omega t),
\end{equation}
with small amplitude $I_0$. In this regime, $g(t)$ and $I(t)$ trace out a 
Lissajous figure in the $(I,g)$ plane after transients decay. The enclosed 
area of this loop, normalized appropriately, provides a measure of 
hysteresis.

We define the \emph{hysteresis prefactor} $C(\omega,K)$ as
\begin{equation}
  C(\omega,K) = \frac{1}{I_0^2}
  \oint g\, \dd I,
\end{equation}
where the contour integral is taken over one period in the steady state. In 
practice, we approximate the loop by a polygon through sampled points 
$(I(t_i),g(t_i))$ and evaluate its area numerically. For fixed $(A,B,\Delta)$ 
in the stable-ringing wedge, we sweep $\omega$ and, optionally, $K$ and 
compute $C(\omega,K)$. 

We expect $C$ to peak near the dominant resonance of the loop and to 
decrease away from it, reflecting the decreasing phase lag and amplitude 
response. A typical frequency dependence $C(\omega)$ for a fixed $K$ in the 
ringing wedge, and a two-dimensional heatmap $C(\omega,K)$ over a small 
region of the wedge, will be shown in Fig.~\ref{fig:hysteresis}.

\begin{figure}[t]
  \centering
  % To be generated by experiments/rfo_hysteresis_frequency.py
  \includegraphics[width=0.48\textwidth]{figures/rfo/hysteresis_panels.png}
  \caption{\textbf{Hysteresis prefactor under sinusoidal drive.}
  Schematic placeholder for (a) hysteresis prefactor $C(\omega,K_0)$ 
  vs frequency for a fixed $K_0$ in the ringing wedge, showing a peak near 
  resonance; and (b) a heatmap of $C(\omega,K)$ in a small region of the 
  wedge, revealing a ridge of enhanced hysteresis near the Ring Threshold. 
  The area of the Lissajous loop in the $(I,g)$ plane quantifies the 
  energetic ``cost'' of maintaining resonance.}
  \label{fig:hysteresis}
\end{figure}

\section{Discussion and Outlook}
\label{sec:discussion}

We have analyzed a minimal delayed plasticity loop, derived from a 
two-variable geometric plasticity model, and shown that its linear 
dynamics admit a clean and quantitative partition of parameter space into 
three regimes: a DC-unstable region where gain exceeds decay ($K>B$), a 
stable overdamped region with monotone relaxation, and a narrow 
stable--ringing wedge bounded by the cubic discriminant $\Delta_{\text{cubic}}=0$ 
and the DC threshold. Within the Pad\'e(1,1)-valid regime, the discriminant-based 
Ring Threshold agrees closely with the onset of underdamped behavior in the 
full delay differential equation.

From a dynamical-systems perspective, the result is modest but sharp: a 
simple delayed second-order system exhibits long-lived, structured transient 
responses only in a well-defined region of delay--gain space. From the 
broader perspective of Resonance Geometry, this wedge provides a concrete 
example of how ``geometric memory motifs''---here instantiated as damped 
oscillatory folds in $g(t)$---require both delay and gain to be tuned near 
the edge of instability. 

In future work we plan to embed this scalar RFO model into higher-dimensional 
networks, examine how geometric plasticity interacts with oscillatory phases 
in coupled oscillators, and explore connections to delay-based reservoir 
computing and neuromorphic photonic implementations. The analytic tools 
developed here (Pad\'e reduction, discriminant-based thresholds, and 
hysteresis quantification) provide a compact language for characterizing 
where in parameter space such networks can support structured, resonant 
memory patterns.

\section*{Code availability}

All analysis and simulation code used in this work, including scripts to 
generate the $(\Delta,K)$ phase diagram, time-series archetypes, threshold 
validation, and hysteresis plots, is available in the public GitHub 
repository \url{https://github.com/justindbilyeu/Resonance_Geometry}. 
In particular, the directory \texttt{scripts/} contains analytic stability 
scans and phase-map plotting tools, while \texttt{experiments/} contains 
time-domain simulation and validation scripts. The corresponding data files 
are stored under \texttt{results/}, and figures under \texttt{figures/}.

\bibliographystyle{apsrev4-2}
% TODO: populate the bibliography file and uncomment \bibliography below.
% Suggested key references to include (representative placeholders):
% - Optoelectronic oscillators and delay systems:
%   Larger2010, Chembo2010, Yanchuk2008, Peil2009
% - Synaptic plasticity and memory in delay systems:
%   ZenkeGerstner2015, GanguliSompolinsky2012
% - Delay-based reservoir computing:
%   Appeltant2011
%
% \bibliography{resonance_geometry_rfo}

\end{document}
