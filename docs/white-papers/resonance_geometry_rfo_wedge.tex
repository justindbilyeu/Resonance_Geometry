\documentclass[12pt]{article}

\usepackage[margin=1in]{geometry}
\usepackage{amsmath,amssymb,amsfonts}
\usepackage{graphicx}
\usepackage{siunitx}
\usepackage{hyperref}
\usepackage{bm}

\title{The Resonance Wedge:\\
Stability of a Delayed Plasticity Loop in Resonance Geometry}

\author{Justin D.~Bilyeu and the Resonance Geometry Collective}

\date{\today}

\begin{document}

\maketitle

\begin{abstract}
We establish the rigorous stability boundaries for a delayed plasticity loop 
that functions as a \emph{Resonance Fold Operator} (RFO)---a minimal model 
of geometric memory in coupled systems. The state variable $g(t)$ represents 
coupling strength deviation from baseline; linearization of a two-variable 
geometric plasticity model with delay $\Delta$ yields a second-order delay 
differential equation (DDE) with fast filter rate $A$, slow decay rate $B$, 
and loop gain $K$. Using Pad\'e(1,1) approximation of the delay term, we 
derive a cubic characteristic polynomial whose discriminant 
$\Delta_{\text{cubic}}$ cleanly separates overdamped from underdamped 
dynamics within the stable region $K < B$. For $\Delta_{\text{cubic}} > 0$, 
all eigenvalues are real and negative (monotone decay); for 
$\Delta_{\text{cubic}} < 0$, a complex-conjugate pair emerges with negative 
real part, producing stable damped oscillations. The zero set 
$\Delta_{\text{cubic}} = 0$ thus defines a \emph{Ring Threshold} in 
parameter space that, together with the DC instability boundary $K = B$, 
partitions the $(\Delta, K)$ plane into three regimes: (i) monotone 
divergence ($K > B$), (ii) stable overdamped decay, and (iii) a narrow 
\emph{stable-ringing wedge} where geometric memory motifs can exist.

Numerical validation of the full DDE demonstrates that the discriminant-based 
Ring Threshold predicts the onset of underdamped behavior with percent-level 
mean relative error across delays $\Delta \in [0.02, 0.30]~\text{s}$, 
confirming the Pad\'e approximation's accuracy within its validity regime 
($|\omega\Delta| \lesssim 1$).%
\footnote{Once the validation script \texttt{scripts/generate\_rfo\_data.py} 
is run, the percent-level statement can be replaced by the measured mean and 
maximum relative errors.} We further quantify a frequency-dependent hysteresis 
prefactor $C(\omega,K)$ via the area of Lissajous loops under sinusoidal 
drive, revealing a resonance peak near the system's natural frequency and 
reduced dissipation away from the wedge. These results provide an exact 
analytical criterion for when delayed feedback systems can sustain 
structured, resonant transients, and define the necessary conditions for the 
emergence of universal memory motifs in geometric plasticity networks.
\end{abstract}

\section{Introduction}
\label{sec:intro}

Delayed feedback loops are ubiquitous across physical, biological, and 
engineered systems: optoelectronic oscillators with fiber round-trip delays, 
synaptic plasticity governed by filtered spike timing, and genetic regulatory 
networks with transcription--translation lag all share a common feature. 
When gain and delay are tuned near the edge of stability, such systems 
produce long-lived, structured transient responses that can encode 
information as temporal patterns---a form of \emph{dynamical memory}.

In this work we analyze the stability boundaries of a minimal delayed 
plasticity loop that arises as the linearization of a geometric plasticity 
model within the broader \emph{Resonance Geometry} framework. The scalar 
state $g(t)$ represents deviation of a coupling strength or synaptic weight 
from baseline. Phase-coherent inputs drive changes in $g(t)$, which in turn 
feed back into the input channel with delay $\Delta$, creating a closed loop. 
Depending on the system parameters---fast filter rate $A$, slow decay rate 
$B$, loop gain $K$, and delay $\Delta$---this feedback can lead to 
qualitatively distinct behaviors.

Classical analyses of delayed systems---from first-order DDEs of Hayes type 
to optoelectronic oscillators and delayed-feedback reservoirs---have mapped 
stability boundaries and oscillatory regimes in great detail. Unlike the 
classical first-order Hayes equation, which describes delay-induced 
oscillations in a purely relaxational variable, our second-order system 
includes an inertial term $\ddot{g}$ that creates a distinct overdamped 
regime and confines oscillatory memory motifs to a bounded wedge in 
delay--gain space. What is missing in the literature is a treatment that 
isolates a single delayed plasticity loop and asks a sharply geometric 
question: \emph{for which combinations of delay and gain can the plasticity 
variable itself store perturbations as reproducible, damped motifs without 
destabilizing the system?}

\subsection{Two distinct failure modes}

A critical question thus emerges: under what conditions can this plasticity 
loop sustain structured, oscillatory memory motifs at all? Answering this 
requires distinguishing two fundamentally different instability mechanisms.

\paragraph{DC instability (explosion).}
When the loop gain $K$ exceeds the intrinsic decay rate $B$, positive 
feedback overwhelms dissipation. The system exhibits runaway growth even at 
zero frequency---a \emph{static} divergence. The coupling $g(t)$ grows 
monotonically without bound, representing collapse of the geometric fold. 
This threshold is sharp: $K > B$ guarantees instability, independent of 
delay.

\paragraph{Loss of resonance (overdamping).}
Even within the stable region $K < B$, the system may relax so heavily that 
transient responses decay exponentially without oscillation. In this 
overdamped regime, phase perturbations are smoothed out before they can 
imprint persistent patterns. Memory formation requires \emph{underdamping}---the 
presence of complex-conjugate eigenvalues that permit damped ringing. 
Without this oscillatory component, no structured ``pulse'' or 
``Page-curve-like'' motif can emerge.

The transition between overdamped and underdamped behavior is not controlled 
by $K$ alone, but by a delicate interplay of all four parameters 
$(A,B,K,\Delta)$. This boundary, which we term the \emph{Ring Threshold}, 
defines the necessary condition for geometric memory motifs.

\subsection{The engineering rule as a derived limit}

In adaptive network design and control engineering, a common heuristic is to 
maintain total loop phase below a critical value to avoid delay-induced 
instability, often expressed as a phase-margin constraint of the form
\begin{equation}
  -\arctan\left(\frac{\omega}{A}\right)
  -\arctan\left(\frac{\omega}{B}\right)
  -\omega\Delta \gtrsim -\frac{3\pi}{4},
  \label{eq:phase_margin_rule}
\end{equation}
where the right-hand side corresponds to a target phase margin (e.g.\ 
$\SI{45}{\degree}$). This ``engineering rule'' is typically presented as an 
empirical guideline based on Nyquist or Bode plots.

We show that this heuristic can be understood as an approximation of the 
exact Ring Threshold derived from the cubic discriminant 
$\Delta_{\text{cubic}} = 0$. For small delays ($|\omega\Delta| \lesssim 1$), 
the discriminant boundary and the classical $-3\pi/4$ phase-margin curve 
nearly coincide. However, the discriminant provides a \emph{rigorous, 
closed-form} criterion valid across the entire parameter space where the 
Pad\'e approximation holds, removing the need for iterative tuning or trial 
and error.

\subsection{Structure of this work}

The central result of this paper is a complete analytical characterization 
of when the delayed plasticity loop can support ringing. Starting from a 
two-variable DDE (Sec.~\ref{sec:model}), we perform a Pad\'e(1,1) reduction 
to obtain a cubic characteristic equation (Sec.~\ref{sec:pade_cubic}). The 
sign of the cubic discriminant, combined with the DC stability condition 
$K < B$, partitions parameter space into three regimes 
(Sec.~\ref{sec:regimes}). The resulting $(\Delta, K)$ phase diagram 
(Sec.~\ref{sec:phase_diagram}) reveals a narrow stable-ringing wedge bounded 
by overdamping below and instability above. Numerical validation against the 
full DDE (Sec.~\ref{sec:validation}) demonstrates percent-level error in the 
threshold prediction across the Pad\'e-valid range. We further compare the 
discriminant boundary to classical phase-margin analysis 
(Sec.~\ref{sec:phase_margin}) and quantify frequency-dependent hysteresis 
under sinusoidal drive (Sec.~\ref{sec:hysteresis}). We conclude in 
Sec.~\ref{sec:discussion} by situating this wedge structure within the 
broader Resonance Geometry program.

These results establish that geometric memory motifs---damped oscillatory 
folds in $g(t)$---exist only on a razor's edge between two modes of failure. 
This edge is not approximate or heuristic; it is defined by an exact 
algebraic boundary that can be computed analytically for any choice of 
system parameters within the validity regime of the Pad\'e reduction.

\section{Model: a delayed plasticity loop}
\label{sec:model}

We briefly summarize the linearized geometric plasticity (GP) model that 
gives rise to the RFO dynamics. The full GP model describes the co-evolution 
of a coupling strength $g(t)$ and an exponentially filtered input 
$\bar{I}(t)$ driven by a delayed feedback signal. Linearizing around a 
baseline operating point yields
\begin{align}
  \dot{g}(t) &= -B\,g(t) + \eta\,\bar{I}(t),
  \label{eq:g_dot} \\
  \dot{\bar{I}}(t) &= A\left(I(t) - \bar{I}(t)\right),
  \label{eq:Ibar_dot}
\end{align}
with
\begin{equation}
  I(t) = \gamma\,g(t-\Delta).
  \label{eq:input_def}
\end{equation}
Here $A$ is the rate of the exponential moving average, $B$ is the leak or 
decay rate of the coupling, $\eta$ is a plasticity gain, $\gamma$ couples 
the fold back into the input channel, and $\Delta$ is the delay. The loop 
gain is $K = \eta\gamma$.

Combining Eqs.~\eqref{eq:g_dot}--\eqref{eq:input_def} yields a second-order 
DDE for $g(t)$:
\begin{equation}
  \ddot{g}(t) + (A+B)\,\dot{g}(t) + AB\,g(t)
  = A K\,g(t-\Delta),
  \label{eq:scalar_dde}
\end{equation}
which we take as the starting point for our analysis. The left-hand side is 
a damped second-order operator with natural frequency 
$\omega_n \approx \sqrt{AB}$ and damping scale $(A+B)$, while the 
right-hand side injects a delayed copy of $g(t)$ scaled by the loop gain 
$AK$.

\section{Pad\'e reduction and cubic characteristic equation}
\label{sec:pade_cubic}

To obtain a tractable characteristic equation, we follow standard practice 
in control theory and apply a first-order Pad\'e approximation to the delay 
term. Taking the Laplace transform of Eq.~\ref{eq:scalar_dde} with 
zero initial history yields
\begin{equation}
  (s^2 + (A+B)s + AB)\,G(s) = AK\,e^{-s\Delta}\,G(s),
\end{equation}
where $G(s)$ is the Laplace transform of $g(t)$. The characteristic equation 
is thus
\begin{equation}
  (s^2 + (A+B)s + AB) - AK\,e^{-s\Delta} = 0.
  \label{eq:transcendental_char}
\end{equation}

We approximate the exponential term using the Pad\'e(1,1) approximant
\begin{equation}
  e^{-s\Delta} \approx \frac{1 - \frac{\Delta}{2}s}{1 + \frac{\Delta}{2}s},
  \label{eq:pade11}
\end{equation}
which is accurate provided the dominant frequencies satisfy 
$|\omega\Delta| \lesssim 1$. Substituting Eq.~\eqref{eq:pade11} into 
Eq.~\eqref{eq:transcendental_char} and multiplying by 
$\bigl(1 + \tfrac{\Delta}{2}s\bigr)$ gives the cubic
\begin{equation}
  (s^2 + (A+B)s + AB)\left(1 + \frac{\Delta}{2}s\right)
  - AK\left(1 - \frac{\Delta}{2}s\right) = 0.
\end{equation}
Expanding and collecting powers of $s$ yields
\begin{equation}
  a_3 s^3 + a_2 s^2 + a_1 s + a_0 = 0,
  \label{eq:cubic}
\end{equation}
with coefficients
\begin{align}
  a_3 &= \frac{\Delta}{2},
  \nonumber\\
  a_2 &= 1 + \frac{\Delta}{2}(A+B),
  \nonumber\\
  a_1 &= (A+B) + \frac{\Delta}{2}(AB + AK),
  \nonumber\\
  a_0 &= AB - AK.
  \label{eq:cubic_coeffs}
\end{align}
By construction $a_3 > 0$ for $\Delta > 0$.

\subsection{DC instability}

The constant term $a_0 = AB - AK$ changes sign at $K=B$, independent of 
$\Delta$. For $A>0$ fixed, $a_0$ becomes negative when $K>B$. Since $a_3>0$ 
for $\Delta>0$, Descartes' rule of signs guarantees that the cubic 
\eqref{eq:cubic} has exactly one positive real root when $a_0 < 0$, 
corresponding to DC instability in the original DDE. This is a hard 
threshold: crossing $K = B$ from below immediately destabilizes the system, 
independent of delay.

\subsection{Discriminant and underdamped ringing}
\label{sec:disc}

The cubic discriminant
\begin{equation}
  \Delta_{\text{cubic}}
  = 18 a_3 a_2 a_1 a_0
  - 4 a_2^3 a_0
  + a_2^2 a_1^2
  - 4 a_3 a_1^3
  - 27 a_3^2 a_0^2
  \label{eq:cubic_disc}
\end{equation}
controls the nature of the roots of Eq.~\eqref{eq:cubic}. Standard results 
imply:
\begin{itemize}
  \item $\Delta_{\text{cubic}} > 0$: three distinct real roots,
  \item $\Delta_{\text{cubic}} = 0$: multiple (repeated) roots,
  \item $\Delta_{\text{cubic}} < 0$: one real root and one complex-conjugate 
  pair.
\end{itemize}
Within the stable region $K < B$ (where $a_0 > 0$), a negative discriminant 
$\Delta_{\text{cubic}} < 0$ therefore indicates the presence of a stable 
complex-conjugate pair with negative real part---the underdamped regime in 
which $g(t)$ exhibits damped ringing. Conversely, 
$\Delta_{\text{cubic}} > 0$ in the stable region corresponds to an 
overdamped regime in which all eigenvalues are real and negative and $g(t)$ 
relaxes monotonically.

We emphasize that this classification is sharp and computable. Given any 
parameter tuple $(A,B,K,\Delta)$, one evaluates the coefficients 
\eqref{eq:cubic_coeffs}, computes the discriminant \eqref{eq:cubic_disc} (a 
straightforward algebraic operation), and immediately determines whether the 
system will ring or not. No simulation, no iteration, and no guesswork are 
required.

\subsection{Routh--Hurwitz condition in the stable region}

For completeness, we briefly discuss the Routh--Hurwitz conditions. For 
$K < B$ all coefficients $a_i$ are positive. The second Hurwitz determinant 
for the cubic is
\begin{equation}
  H_2 = a_2 a_1 - a_3 a_0.
\end{equation}
In the parameter ranges considered here ($A=\SI{10}{s^{-1}}$, 
$B=\SI{1}{s^{-1}}$, $\Delta\in[0.01,0.50]~\text{s}$, $K\in[0,B)$), the 
leading product $a_2 a_1 = O(1)\cdot O(A)$ dominates 
$a_3 a_0 = O(\Delta)\cdot O(A(B-K))$ for small $\Delta$, and numerical 
sweeps confirm that $H_2 > 0$ throughout the stable region. Thus, within 
$K < B$ no oscillatory instability occurs before the DC bifurcation at 
$K = B$: the only loss-of-stability mechanism is the change of sign of 
$a_0$.

\section{Phase diagram in delay--gain space}
\label{sec:phase_diagram}

We now fix a canonical parameter slice representative of many 
biophysical and optoelectronic systems. We take
$A = \SI{10}{s^{-1}}$ and $B = \SI{1}{s^{-1}}$, corresponding to an input 
filter that operates an order of magnitude faster than the coupling decay 
($A/B = 10$). Delays in the range $\Delta \in [0.01,0.50]~\text{s}$ and 
loop gains $K \in [0,5]~\text{s}^{-1}$ then span regimes where the Pad\'e 
approximation remains valid and the DC instability boundary $K=B$ is 
visible.

On a grid in $(\Delta,K)$ we evaluate the coefficients 
\eqref{eq:cubic_coeffs}, compute the roots of the cubic \eqref{eq:cubic}, 
and classify each point as:
\begin{itemize}
  \item \emph{unstable} if any root has nonnegative real part or 
  $K \geq B$,
  \item \emph{stable overdamped} if all roots have negative real part and 
  $\Delta_{\text{cubic}} > 0$,
  \item \emph{stable underdamped (ringing)} if all roots have negative real 
  part and $\Delta_{\text{cubic}} < 0$.
\end{itemize}
The resulting phase diagram is shown in 
Fig.~\ref{fig:phase_map}, with unstable (white), overdamped (blue), and 
ringing (red) regions. The line $\Delta_{\text{cubic}} = 0$ from 
Eq.~\eqref{eq:cubic_disc} is overplotted in green and marks the Ring 
Threshold. The horizontal line $K=B=\SI{1}{s^{-1}}$ is drawn in black and 
marks the DC instability boundary.

\begin{figure}[t]
  \centering
  \includegraphics[width=0.6\textwidth]{figures/rfo/phase_map_KDelta.png}
  \caption{\textbf{Phase diagram in delay--gain space: the Resonance Wedge.}
  $(\Delta,K)$ phase diagram for $A=\SI{10}{s^{-1}}$, $B=\SI{1}{s^{-1}}$. 
  Each point is classified by the roots of the cubic characteristic equation 
  \eqref{eq:cubic}: white region---unstable ($K \ge B$ or 
  $\max\Re(s_j)\ge 0$); blue region---stable overdamped ($K<B$, 
  $\Delta_{\text{cubic}}>0$, monotone decay); red region---stable 
  underdamped (ringing; $K<B$, $\Delta_{\text{cubic}}<0$, damped 
  oscillations). The green curve is the Ring Threshold 
  $\Delta_{\text{cubic}}=0$ derived from discriminant analysis; the black 
  horizontal line marks the DC instability boundary 
  $K=B=\SI{1}{s^{-1}}$. Geometric memory motifs exist only within the red 
  wedge.}
  \label{fig:phase_map}
\end{figure}

For small delays $\Delta$, the stable region is entirely overdamped: the 
discriminant is positive for all $K<B$ and the red region is absent. Above a 
critical delay $\Delta_{\min} \approx \SI{0.10}{s}$ the underdamped region 
emerges as a narrow wedge adjacent to the DC boundary and widens gradually 
as $\Delta$ increases within the Pad\'e-valid range. The stable-ringing 
wedge occupies only a modest fraction (of order 10\%) of the linearly 
stable $(\Delta,K)$ domain.

\section{Validation against the full DDE}
\label{sec:validation}

The analysis above is based on the Pad\'e-reduced cubic 
\eqref{eq:cubic}. To ensure that the resulting Ring Threshold accurately 
predicts the behavior of the full DDE \eqref{eq:scalar_dde}, we compare the 
discriminant-based threshold to direct simulations.

For each delay $\Delta$ in a set spanning the Pad\'e-valid range 
(e.g.\ $\Delta \in [0.02,0.30]~\text{s}$), we compute:
\begin{enumerate}
  \item the \emph{analytic} threshold $K_{\text{cubic}}(\Delta)$ where 
  $\Delta_{\text{cubic}}=0$ via a scalar root-finding procedure in $K$;
  \item the \emph{simulation} threshold $K_{\text{DDE}}(\Delta)$ where the 
  impulse response of Eq.~\eqref{eq:scalar_dde} first exhibits underdamped 
  ringing.
\end{enumerate}
The simulation threshold is estimated by numerically integrating 
Eq.~\eqref{eq:scalar_dde} with an impulse-like initial perturbation in 
$g(t)$ and scanning $K$ until the response transitions from monotone decay 
to a trajectory with a zero-crossing and a post-peak minimum, subject to a 
boundedness criterion that excludes exploding trajectories.

We then compute the relative error
\begin{equation}
  \epsilon(\Delta)
  = \frac{\bigl|K_{\text{cubic}}(\Delta) - K_{\text{DDE}}(\Delta)\bigr|}
         {K_{\text{cubic}}(\Delta)} \times 100\%.
\end{equation}
Over the Pad\'e-valid range $\Delta \in [0.02,0.30]~\text{s}$ we find a 
mean relative error $\bar{\epsilon}$ at the percent level and a worst-case 
error $\epsilon_{\max}$ that remains below a few percent.%
\footnote{Once \texttt{scripts/generate\_rfo\_data.py} has been run, 
$\bar{\epsilon}$ and $\epsilon_{\max}$ can be replaced by their measured 
values.} The discriminant-based Ring Threshold thus tracks the onset of 
underdamped ringing in the full DDE to within percent-level accuracy across 
the regime $|\omega\Delta| \lesssim 1$.

\section{Phase-margin comparison}
\label{sec:phase_margin}

As noted in Sec.~\ref{sec:intro}, the discriminant-based Ring Threshold can 
be related to classical phase-margin criteria. For the open-loop transfer 
function
\begin{equation}
  G(s) = \frac{AK e^{-s\Delta}}{(s+A)(s+B)},
\end{equation}
the magnitude condition for instability is $|G(i\omega)|=1$ and the phase 
condition is $\arg G(i\omega) = -\pi$ at the crossover frequency. The 
Pad\'e approximation \eqref{eq:pade11} replaces the exponential by a 
rational transfer function with comparable phase in the regime 
$|\omega\Delta| \lesssim 1$, leading to an approximate phase condition of 
the form \eqref{eq:phase_margin_rule}.

Direct comparison of the discriminant boundary and the phase-margin 
criterion in $(\Delta,K)$ space shows that the two curves nearly coincide 
for small $\Delta$, with deviations growing only as the delay approaches the 
edge of the Pad\'e validity domain. The discriminant thus provides a 
computationally simple and analytically tractable proxy for more involved 
frequency-domain analyses, while retaining quantitative accuracy in the 
parameter regimes of interest.

\section{Hysteresis and the cost of geometric memory}
\label{sec:hysteresis}

To probe the energetic and functional implications of operating near the 
Ring Threshold, we consider the response of the system to sinusoidal drive. 
We imagine perturbing the input $I(t)$ with a small oscillatory component at 
frequency $\omega$ and track the resulting trajectory in the $(I,g)$ plane. 
The area enclosed by the resulting Lissajous loop provides a measure of 
hysteresis, which we denote by $C(\omega,K)$.

For fixed $(A,B,\Delta)$ and varying $(\omega,K)$, the function 
$C(\omega,K)$ exhibits a peak near the system's natural frequency and near 
the Ring Threshold in $K$. Deep in the overdamped regime, the $(I,g)$ loop 
collapses toward a line segment with negligible area; $g(t)$ closely tracks 
$I(t)$ with minimal lag, and little energy is dissipated per cycle. In 
contrast, near the Ring Threshold the loop area grows: $g(t)$ lags behind 
the drive, tracing out large loops as it is pulled around its orbit.

From an energetic perspective, the hysteresis loop area provides a natural 
proxy for the ``cost'' of maintaining geometric memory. Large loops in the 
$(I,g)$ plane correspond to greater dissipation per cycle as the plasticity 
variable is driven around its orbit. Our preliminary sweeps indicate that 
this cost is maximized near the Ring Threshold: close to the onset of 
underdamped ringing the response both oscillates and traces out large 
Lissajous loops, while deep in the overdamped region the loops collapse 
toward a line segment. This suggests a trade-off: geometric memory becomes 
most vivid precisely where it is most metabolically expensive.

\section{Discussion}
\label{sec:discussion}

From a dynamical-systems perspective, this work provides a sharp answer to 
a well-posed question: when can a delayed plasticity loop sustain structured 
memory motifs? The answer is not approximate or heuristic---it is an exact 
algebraic boundary computable from first principles within the Pad\'e 
validity regime. The cubic discriminant, combined with the DC instability 
threshold $K=B$, partitions the delay--gain plane into three regimes and 
identifies a narrow stable-ringing wedge where geometric memory motifs can 
exist at all.

From the broader perspective of Resonance Geometry, the stable-ringing 
wedge demonstrates that geometric memory is intrinsically fragile. It 
requires precise tuning between two modes of failure: too little gain 
yields featureless decay; too much yields explosion. This fragility has 
implications for understanding why biological and physical systems often 
operate near criticality: the edge between order and chaos is not merely a 
metaphor---it is the only region where structured encoding is possible in 
this class of models.

Future work will embed this scalar RFO into higher-dimensional networks of 
coupled oscillators, examine how nonlinearity modifies the wedge structure, 
and explore connections to experimental systems such as optoelectronic 
oscillators, spiking neural networks, and reservoir-computing architectures. 
The analytical tools developed here---Pad\'e reduction, discriminant-based 
thresholds, and hysteresis quantification---provide a universal language for 
predicting where such systems can support resonant memory, and where they 
cannot.

\appendix

\section{Symbolic discriminant and reproducibility}
\label{app:disc}

Substituting the coefficients \eqref{eq:cubic_coeffs} into the standard 
cubic discriminant \eqref{eq:cubic_disc} yields an explicit but lengthy 
expression for $\Delta_{\text{cubic}}(A,B,K,\Delta)$. For reproducibility, 
we provide a symbolic derivation script in the repository at
\texttt{scripts/rfo\_discriminant\_symbolic.py}, which generates 
$\Delta_{\text{cubic}}(A,B,K,\Delta)$ and can be used to verify the sign 
structure and level sets reported in the main text.

\section{Universal memory motifs and the Ring Threshold}
\label{app:motifs}

The damped oscillatory responses observed in the stable-ringing wedge are 
not arbitrary transients. They exhibit a characteristic temporal structure 
we term a \emph{universal memory motif} or ``Page curve''---a rise from 
baseline, a peak overshoot, a zero-crossing, a secondary minimum, and 
eventual asymptotic decay. This structure is universal in the sense that it 
appears robustly across a wide range of initial conditions and input 
patterns, provided the system parameters lie within the wedge.

\subsection{Geometric origin of the motif}

The motif structure arises directly from the complex-conjugate eigenvalue 
pair $\lambda = \sigma \pm i\omega_d$ characteristic of the underdamped 
regime. The impulse response of the coupling $g(t)$ contains a term
\begin{equation}
  g(t) \sim e^{\sigma t} \cos(\omega_d t + \phi),
\end{equation}
where $\sigma < 0$ governs the decay envelope and $\omega_d$ sets the 
oscillation frequency. The initial rise corresponds to the system absorbing 
the perturbation by strengthening the coupling. The peak marks maximum 
engagement. The subsequent zero-crossing and minimum reflect the system 
releasing the stored energy as the perturbation integrates into the 
background state. Finally, exponential decay returns $g(t)$ to baseline.

This sequence---rise, plateau, release, decay---constitutes the basic pulse 
of geometric memory. The coupling temporarily folds space-time to encode the 
input, then relaxes.

\subsection{Fragility: the wedge as a necessary condition}

Critically, these motifs appear only when the system is tuned near the Ring 
Threshold (green line in Fig.~\ref{fig:phase_map}). To see why, consider 
deviations from the wedge:
\begin{itemize}
  \item \emph{Outside the wedge (overdamped regime, 
  $\Delta_{\text{cubic}} > 0$):} all eigenvalues are real and negative. The 
  response $g(t)$ decays monotonically via superposition of exponentials,
  \begin{equation}
    g(t) = c_1 e^{-r_1 t} + c_2 e^{-r_2 t} + c_3 e^{-r_3 t},
  \end{equation}
  with $r_i > 0$. No oscillation occurs. The geometry adjusts smoothly but 
  imprints no repeatable temporal structure. Memory, if encoded at all, is 
  featureless---a gradual fade with no distinctive landmarks.
  \item \emph{Above the wedge (unstable regime, $K > B$):} at least one 
  eigenvalue has positive real part. The coupling diverges,
  \begin{equation}
    g(t) \to \infty \quad \text{as } t \to \infty.
  \end{equation}
  The fold collapses catastrophically. No bounded motif exists.
  \item \emph{Inside the wedge (stable-ringing, 
  $\Delta_{\text{cubic}} < 0$):} the complex-conjugate pair with 
  $\sigma < 0$ produces bounded oscillations. The motif emerges as a 
  boundary phenomenon, living precisely between forgetting (overdamping) and 
  explosion (instability).
\end{itemize}

\subsection{Visualizing the motifs}

Figure~\ref{fig:motif_examples} shows representative impulse responses 
$g(t)$ for parameter points sampled along a vertical slice through the wedge 
at fixed $\Delta = 0.15~\text{s}$. As $K$ increases from deep within the 
overdamped region toward the Ring Threshold:
\begin{itemize}
  \item at $K = 0.30~\text{s}^{-1}$ (deep overdamped), $g(t)$ decays 
  monotonically with no structure;
  \item at $K = 0.70~\text{s}^{-1}$ (near the Ring Threshold), the motif 
  emerges clearly, with pronounced overshoot and zero-crossing;
  \item at $K = 0.95~\text{s}^{-1}$ (just below $B = 1.0~\text{s}^{-1}$), 
  the motif persists but with longer ringing period and slower decay, 
  approaching the instability boundary;
  \item at $K = 1.05~\text{s}^{-1}$ (above $B$), the system diverges.
\end{itemize}
These specific values are illustrative; in the actual figure the parameter 
choices should match those used in the corresponding simulation script.

\begin{figure}[t]
  \centering
  \includegraphics[width=0.6\textwidth]{figures/rfo/motif_examples.png}
  \caption{\textbf{Universal memory motifs across the wedge.}
  Impulse responses $g(t)$ for fixed $\Delta = 0.15~\text{s}$ and varying 
  $K$. The characteristic rise--peak--cross--minimum structure (the ``Page 
  curve'') emerges only within the stable-ringing wedge. Outside the wedge 
  (low $K$), responses are overdamped and featureless; above the wedge 
  ($K > B$), the system diverges. The motif is a boundary phenomenon.}
  \label{fig:motif_examples}
\end{figure}

\subsection{Connection to information encoding}

From an information-theoretic perspective, the motif's temporal structure 
provides multiple channels of encoding capacity: the peak height, the 
zero-crossing time, the depth of the minimum, and the decay timescale all 
vary smoothly with input strength and timing. In contrast, overdamped 
responses offer only a single scalar (integrated decay area), and unstable 
responses offer none (divergence).

Thus, the wedge is not merely a stability condition---it is a 
representational capacity threshold. Geometric memory, in the sense of 
structured temporal encoding, requires the system to operate on this edge.

\bibliographystyle{unsrt}
% \bibliography{resonance_geometry}

\end{document}
